\tableofcontents
% \vspace{2cm} %Add a 2cm space

\begin{abstract}
	Plan:
	\begin{enumerate}
		\item Courbes (plan + espace)
		\begin{itemize}
			\item étude local
			\item étude global
		\end{itemize}
		\item surfaces dans $\mathbb{R}^3$
	\end{enumerate}
\end{abstract}
 
	\section{Courbes}

	\emph{Lesson 1}

	% definition of a curve and regular curve
	\theoremstyle{definition}
	\begin{definition}[Courbe et Courbe Régulière]
		\leavevmode
		\begin{enumerate}
			\item Une Courbe Paramètre dans $\mathbb{R}^3$ est une function $c:I\rightarrow \mathbb{R}^n$ où $I$ est un intervalle de $\mathbb{R}$ et $c$ est lisse ($c$ est infiniment différentielle, $ c \in C^\infty$).
			$$I\ni t\mapsto c(t)\in \mathbb{R}^3,$$
			$t$ -- paramètre.
			\item Une courbe paramètre est régulièrement si
			$$\dot{c}(t) = \frac{\diff}{\diff t}c(t)\neq 0,$$
			pour tout $t\in I$.
		\end{enumerate}
	\end{definition}

	Si une courbe est régulière, $c(t)\neq \ct{const}$. $\dot{c}(t)$ désigne la tangente à la courbe en $c(t)$.

	Chaque régulière courbe est tangente à la ligne.

	\begin{definition} La trace d'une courbe paramètre $I\ni t \mapsto c(t)\in \mathbb{R}^n$ est image:
		$$\{c(t)\ |\ t\in I\} \subset \mathbb{R}^n.$$
	\end{definition}

	Une cure paramètre est plus que sa trace.

	La courbe $\R\ni t \mapsto \left( \begin{array}{c} t^3 \\ 0 \end{array} \right) \in \mathbb{R}^2,$
	$\trs = \{ \left( \begin{array}{c} x \\ 0 \end{array} \right)\ |\ x\in \mathbb{R} \}$. Et la courbe $R\ni t \mapsto \left( \begin{array}{c} t \\ 0 \end{array} \right) \in \mathbb{R}^2$ a la même trace!

	$$ \dot{c}_1(t) = \left( \begin{array}{c} 3t^2 \\ 0 \end{array} \right),\ mais\ \dot{c}_2(t) = \left( \begin{array}{c} 1 \\ 0 \end{array} \right).$$

	\begin{definition} Si $I\ni t \mapsto c(t)\in \mathbb{R}$ est une courbe paramètre, $J\subset \mathbb{R}$ -- une intervalle et $\varphi: J\rightarrow I$ une function lisse t.q. $\varphi^{-1}: J\rightarrow I$ est également lisse, on disque(?):
		$$J\ni t \mapsto c^2(t) = c\circ\varphi(t) \in \mathbb{R}^n,$$
		est une reparamétrisation  de  $c$.
	\end{definition}

\begin{remark}  
	$\dot{\tilde{c}}(t)=\dot{c}\circ\varphi(t)*\dot{\varphi}(t)$. Donc,
	$\tilde{c}$ - régulière $\Longleftrightarrow$ $c$ est régulière.
	$$\frac{d}{ds}\varphi^{-1}(s)=\frac{1}{\dot{\varphi}\circ\varphi^{-1}(s)}\neq 0$$

	$\varphi: J\rightarrow I$ est un difféomorphisme comme $\dot{\varphi}\neq0$, on a

	$$\left\{\begin{array}{rl}
	\mbox{soit}\ \dot{\varphi}(t)>0, & \mbox{pour tout } t\in J \\ 
	\mbox{soit}\ \dot{\varphi}(t)<0, & \mbox{pour tout } t\in J 
	\end{array}\right.,$$

	$$\left\{\begin{array}{rl}
	\varphi\mbox{ est } \nearrow \\ 
	\varphi\mbox{ est } \searrow
	\end{array}\right..$$

	Si $\varphi$ est $\nearrow$ on dit une la reparamétrisation conserve le sens de parcours (l'orientation).
	Si $\varphi$ est $\searrow$, la reparam inverse le sens de parours.
\end{remark}

	\begin{definition}
		\leavevmode
		\begin{enumerate}
			\item Une \textdemp{courbe} est une Classe d'Equivalence de Courbes Paramètre pour la relation:
			$$c\sim \tilde{c}\Longleftrightarrow\tilde{c}\mbox{ est une reparamétrisation de }c$$
			\item Une \textdemp{courbe orientée} est une classe d'equivalence des courbes paramètre pour:
			$$c\sim \tilde{c}\Longleftrightarrow\tilde{c}\mbox{ est une reparamétrisation préservante la sens de parcours de }c$$
		\end{enumerate}
	\end{definition}

	\begin{definition}
		Si $c$ est une courbe paramètre t.q. $|\dot{c}(t)|=1$ pour tout $t\in I$. On dit que c'est paramètre pur sa longueur d'arc. 
	\end{definition}

	\begin{proposition}
		Si $I\ni t\mapsto c(t)\in \mathbb{R}^n$ est une courbe paramètre régulière il existe une reparamétrisation de $c$ sa long d'arc:
		$$J\ni s\mapsto \tilde{c}(s)=c\circ\varphi(s)\in \mathbb{R}^n$$
		$$|\dot{\tilde{c}}(s)|=1 \mbox{ pour tout } s\in J.$$
	\end{proposition}

	\begin{lemme}
		Si $\begin{array}{rl} J_1\ni s &\mapsto \tilde{c_1}(s)\\ J_2\ni s &\mapsto \tilde{c_2}(s) \end{array}$ sont 2 paramètre de par long d'arc de la meme courbe $|\dot{c_1}(s)| = 1 = |\dot{c_2}(s)|$.
			alors $c_2(s)=c_1(s_0\pm s)$, pour un $s_0\in \mathbb{R}$ et si $c_1$ et $c_2$ ont un pos le meme suis de parcours. Si $c:[a,\ b]\rightarrow \mathbb{R}^n$ est une courbe paramètre sa longueur est:
			$$ L[c] = \int_{a}^{b} |\dot{c}(t)|\, \dd{}t$$
			$$l =\int_{0}^{t} |\dot{c}(u)|\dd{u} = t$$ % is a curve length from point u(0) to u(t), it equals t. There is just a single param like that up to a constant.
	\end{lemme}


		\begin{definition}
			Une courbe paramétrique $c:R\rightarrow R^d$ est appelée \textsc{Périodique} de période $p$, si $c(t+p)=c(t),\ \forall t\in R$.
		\end{definition}

		\begin{definition}
			Une courbe fermée et appeler une \textsc{Courbe Fermée Simple} s'il existe une parametrisation régulière, périodique de période $p$ et si: $c_{[0, p)}$ est injectif.
		\end{definition}

		\begin{definition}
			$c\in C^\infty(I,\ R^2)$ est appelée \textsc{Courbe Plane}.
		\end{definition}

		\begin{definition}
			Soit $c$ une courbe paramètre par longueur d'arc (donc une courbe de vitesse 1) (donc $||\dot{c}(t)||=1$). Son champs normale est définie par:
			$$N(T):=\dot{c}^\perp(t),\ t\in I$$
		\end{definition}

		\begin{remark}
			$N(t)=\left(\begin{array}{cr} 0 & -1 \\ 1 & 0\end{array}\right)\dot{c}(t)$. $N$ depend de l'orientation de la courbe.
		\end{remark}

		Pour chaque $t$ le système ${\dot{c},\ N(t)}$ est un base orthonormée direct de $R^2$.

		\begin{lemme}
			Soit une courbe vitesse 1, $N$ son champs normals alors $\ddot{c}(t)$ est parallèle a $N(t)$.
		\end{lemme}
		\begin{proof}
			Idee $||\dot{c}(t)||=1,\ \forall t \Longleftrightarrow \ddot{c}(t)\perp\dot{c}(t)$.
		\end{proof}

		\begin{definition}
			Soit $c\in C^\infty(I,\ R^2)$ une courbe plane de vitesse 1, alors $\ddot{c}(t)=κ(t)N(t)$, avec $κ(t):=\expval{\ddot{c}(t),\ N(t)}$.
			$κ(t)$ - scalar.
	
			Alors $κ\in C^\infty (I,\ R)$ et $κ$ est appelé la courbure de $c$ ($κ(t)$ la courbure du point $c(t)$)
		\end{definition}

		\begin{theorem}{Formulas de Frenet}
			Soit $c\in C^\infty(I,\ R^2)$ une courbe de vitesse 1.
	
			Soit $\begin{array}{c}T(t):=\dot{c}(t),\\ N(t):=T^\perp (t) \end{array}$, $\{T(t),\ N(t)\}$ - le systeme ortogonale vecteur. Est appellé le \textsc{Repére de Frenet}, ou \textsc{Base de Frenet}. 
			
			\textsc{Formules de Frenet}:
			$$\begin{array}{rcl}\dot{T}(t)&=&κ(t)N(t)\\ \dot{N}(t)&=&-κ(t)T(T)\end{array}$$
		\end{theorem}

		\begin{remark}
			$$\frac{\diff}{\diff\, t}\left(
			\begin{array}{c}
				T\\N\end{array}\right) = \left(\begin{array}{rc}0 & κ\\
				-κ & 0
			\end{array}
			\right) \left(\begin{array}{c}T\\N\end{array}\right)$$
		\end{remark}

		\begin{lemme}
			Soit $c:C^\infty([a,\ b],\ R^2)$ une courbe plane de vitesse, alors il existe $\nu\in C^\infty([a,\ b],\ R)$ t.q. $\dot{c}(t)=(\cos\nu(t),\ \sin\nu(t))$
		\end{lemme}

		\begin{definition}
			Soit $c\in C^\infty(R,\ R^2)$ une courbe plane, périodique de période L et de vitesse 1. En particulier régulière. Soit $\nu\in C^\infty(R,\ R)$.
			Talque $\dot{c}(t)=(\cos \nu (t),\ \sin\nu(t))$ (an dit: une angle de la tangente).
	
			On define Le Nobre de rotation de la tangente de $c$: $n_c:=\frac{1}{2\pi}(\nu(c)-\nu(o))$
		\end{definition}
		% from $... R^3 ...$ to $... \mathbb{R}^3 ...$ 
		% find: \$([^\n\$]*)R([^\n\$]*)\$
		% replace: $$1\mathbb{R}$2$
\begin{rappel}
$c\in C^\infty(I; \R^2)$ reguliere. Alors $\exists\nu\in C^\infty (I; )$ t.q. $\dot{c}(t)=(\cos \nu(t), \sin\nu(t))$. On définie le \textsc{Nombre de Rotation de la Tangente} pour une courbe periodique de période $L$:
		$$n_c:=\frac{1}{2\pi}(\nu (L)-\nu(0))$$
\end{rappel}
	
		\begin{lemme}
			Soient $c_1, c_2\in C^\infty (\R; \R^2)$ deux courbes périodiques de période $L$, paramètre par longueur d'arc $S$: $c_1=c_2\circ\varphi$ avec $\varphi>0 $ alors:
			$$n_{c_1}=n_{c_2}$$
			Si $\dot{\varphi}<0$ alors
			$$n_{c_1}=-n_{c_2}$$
		\end{lemme}
		\begin{remark}
			Le nombre de rotation de la tangente est donc invariant par rapport à une reparamétrisation que preserve l'orientation.
		\end{remark}
		\begin{proof}
			On avait vu que $\varphi(t)=\pm t+t_0$ donc $\dot{\varphi}>0$ $=>$ $\varphi (t)=t+t_0$. Soit $\nu_2$ t.q. $\dot{c}_2(t)=(\cos\nu_2(t), \sin\nu_2(t))$ alors pour $\nu_1:=\nu_2\circ\varphi$ on a que $\dot{c}_1(t)=(\cos\nu_1(t), \sin\nu(_1t))$. Soit $\bar{\nu}_1(t):=\nu_1(t+L)$ on a que $\dot{c}(t)=(\cos\bar{\nu}_1(t), \sin\bar{\nu}(_1t))$ car $c_1(t)=c_1(t+L)$.
			\begin{align}
				2\pi(n_{c_2}-n_{c_1}) & = (\nu_2(L)-\nu_2(0))-(\nu_1(L)-\nu_1(0))
				& =(\nu_2(L-t_0)-\nu_2(-t_0)) - (\nu_1(L)-\nu_1(0))
				& = ...
				=0
			\end{align}
		\end{proof}

		\begin{theorem}
			Sait $c$ une courbe plane périodique de période $L$ et paramètre par longueur d'arc. Soit $\kappa$ la courbure de $c$ alors
			$$n_c=\frac{1}{2\pi}\int_0^L\kappa(t)\dd{t}$$
		\end{theorem}
		
		\begin{remark}
			En particulier $\int\limits_0^L\kappa(t)\dd{t}\in 2\pi\mathbb{Z}$
		\end{remark}
		\begin{proof}
			Soit $\nu\in C^\infty(\R, \R)$ une fonction angle pour la tangente, c.à.d. $\dot{c}(t)=(\cos \nu(t), \sin\nu(t))$. $\ddot{c}(t)=\kappa(t)\dot{c}^\perp (t)$ donc $\kappa(t)=\expval{\ddot{c}(t), \dot{c}^\perp (t)}$ ou $\ddot{c}(t)=\dot{\nu}(t)(-\sin\nu(t), \cos\nu(t))$ et $\dot{c}^\perp(t)=(-\sin\nu(t),\cos\nu(t))$
			donc $<\ddot{c}(t), \dot{c}^\perp(t)>=\dot{\nu}(t)=\kappa(t)$ ou
			$$n_c = \frac{1}{2\pi}(\nu(L)-\nu(0))=\frac{1}{2\pi}\int_0^L \dot{\nu}(t)\dd{t}=\frac{1}{2\pi}\int_0^L\kappa(t)\dd{t}.$$
		\end{proof}

		\begin{theorem}[Hopf. Turning tangent theorem]
			Une courbe plane fermée simple a un nombre de rotation (de la tangente) 1 ou $-1$.
		\end{theorem}

		Nombre de rotation $n=\frac{1}{2\pi}\int\limits_0^L\kappa(t)\dd{t}=\frac{1}{2\pi}(\nu(L)-\nu(0))$. $c(t+l)=c(t)$ $c(t)=(\cos \nu (t), \sin \nu(t)),\ \dot\nu=\kappa$
		\begin{remark}
			On avait inclu dans la défini de fermée simple qu'il n'ya pas de point singulier.
		\end{remark}
	
		Pour la preuve on aura besoin du lemme de recouvrement.

		\begin{definition}
			Sait $X\subset \R^d$ et $x_0\in X$ On dit que $X$ est \textsc{Étoile} par rapport à $x_0$, ($X$ is star shaped). Si pour chaque $x\in X$ le segment de droite entre $x_0$ et $x$ est contenu dans $X$. C'est dire $\forall x$ $\{x_0{1-t}+xt, t\in[0,1]\}\subset X$
		\end{definition}

		\begin{lemme}{De Recouvrement}
			Soit $X\subset \R^d$ étoilé par rapport à $x_0$ et soit
			$$e: X\rightarrow S^1=\{(x,y)\in\R^2, x^2+y^2=1\} \text{---une application continue}$$
	
			Alors in existe une application \underline{continue} $\nu: X\rightarrow \R$ t.q. $e(x)=(\cos\nu(x), \sin\nu(x))$. $\nu$ est unique sous la condition $\nu(x_0)=\nu_0$.
		\end{lemme}

		\begin{proof}
			\underline{Cas} ou $e: X\rightarrow S^1$ n'est pas surjective. Supposons qu'il existe $\varphi_0\in \R$ t.q. $(\cos\varphi_0, \sin\varphi_0)\notin e(X)$. $e(X)=\{z; z=e(x), x\in X\}$.
			La fonction $\psi:(\varphi_0, \varphi_0+2\pi)\rightarrow S^1\\\{(\cos\varphi_0, \sin\varphi_2\}$ est un homéomorphisme.
			On $\nu=\psi^{-1}\circ e$ donc $\nu$ est continue.
	
			\underline{Cas} $e(X)=S^1$. Dans le cas $d=1$, $X=[0,1]$, $x_0=0$ on a démontré le théorème ($e=\dot{c}$ dériver d'une courbe) %img 4planes
	
			\underline{Cas} $d>1$. Soit $x\in X$. On defini $e_x:[0,1]\rightarrow S^1$, $e(x)(t)=e(tx+(1-t)x_0)$. %img a star
			On sait qu'il existe $\nu_x:[0,1]\rightarrow \R$ continue t.q. $e_x(t)=(\cos\nu_x(t),\sin\nu_x(t))$ de $\nu_x(t)=\nu(tx+(1-t)x_0)$ donc $\nu(x)=\nu_x(1)$ donc
			$e(x)=e_x(1)=(\cos\nu_x(1), \sin\nu_x(1))$ is est e a de monte que $\nu_x(1)$ est continue en $e$.
	
			Soit $\eps>0$ et $0=t_0<t_1<t_2<...<t_n=1$ une partition t.q. $e_x|_{[t_j, t_{j+1}]}\subset U_h,\ H\in\{1,2,3,4\}$. Soit $y$ t.q. $\norm{e_x(t)-e_y(t)}<\eps,\ \forall t\in [0,1]$. Si $\eps$ est suffisent petit. $e_y|_{[t_j, t_{j+1})}\subset U_h$. Par example dans le cas $h=4$ on aura
			\begin{align}
				\nu_x(t) &=\arctan\left(\frac{e_x^2(t)}{e_x^1(t)}\right)
				\nu_y(t) &=\arctan\left(\frac{e_y^2(t)}{e_y^1(t)}\right)		
			\end{align}
			$e=(e^1, e^2)$
		\end{proof}

		\begin{proof}{du théorème de Hopf}
			%%img kidney
			Soit $c$ une une paramétrisation de vitesse 1 de période $L$. Sait $x_0:=\max\{c^1(t); t\in [0, l]\}$. Soit $p=\{(z_1, z_2); z_1=x_0\}\cap C(\R)$
			Soit la paramétrisation t.q. $c(0)=p$. $G=p+\R(1,0)$. $C(\R)\cap G$ est à gauche de $p$.
			Soit $X=\{(t_1, t_2): 0\leq t_1\leq t_2 \leq L\}$ %img trin
			$X$ est étoilé par rapport à $(0,0)$. On considère $c:X\rightarrow S^1$
			Formula after an image.
			$$c(t_1,t_2)=\left\{ \begin{array}{cr}\frac{c(t_1)-c(t_1)}{||c(t_1)-c(t_1)||} & t_2>t_1 \\ \dot c(t) & t_2=t_1=t \\ -\dot c(0) & (t_1, t_2)=(0,L)\end{array}\right.$$
	
			Alors $e\in C^0(x, S^1)$, en effet $c\in C^\infty.$ $c(t_2)=c(t_1)+\dot c(t_1)(t_2-t_1)+o(|t_2-t_1|)$
	
			$$\frac{c(t_1)-c(t_1)}{||c(t_1)-c(t_1)||}=\frac{(t_2-t_1)(\dot c(t_1)-o(1))}{||(t_2-t_1)(\dot c(t_1)-o(1))||}\to \frac{\dot c(t_1)}{||\dot c(t_1)||}=\dot c(t_1)$$
			$$t_2\to t_1$$
			$$\frac{c(L-\eps)-c(0)}{||c(L-\eps)-c(0)||}=\frac{c(-\eps)-c(0)}{c(-\eps)-c(0)}=\frac{-\eps(\dot c(0)+o(1))}{||-\eps(\dot c(0)+o(1))||}\to -\dot c(0)$$
			$$\eps\to(down) +0+$$
	
			De plus X est étoilée par rapport à $(0,0)$. Donc il exist $\nu\in C^0(X)$ t.q. $e(t_1, t_2)=(\cos \nu(t_1,t_2), \sin \nu(t_1,t_2))$. Pour de nombre de rotation de ( la tangente de) on a:
			$$2\pi n_c=\nu(L,L)-\nu(0,0)=\nu(L,L) - \nu(0,L)+\nu(0,L)-\nu(0,0)$$
	
			%img nut.ai
	
			(droite $\perp$ à $\dot c(0)$) $x_0=\max \{ c^{(1)},\ t\in [0, L]\}$ $(1,0)\not\in im([0,1]\ni t\mapsto e(0,t))$ car en $c(0), t\mapsto x(t)$ est maximal, donc $im([0,1]\ni t\mapsto \nu(0,t))\subset (0,2\pi)+2\pi k$ (car facile du lemme du recouvrement).
	
			$e(0,L)=-\dot c(0)=(0,-1)$ donc $\nu (0,L)=\frac{3\pi}{2}+2\pi k$ de $\nu(0,0)=\frac{\pi}2+2\pi k $ donc $\nu(0,L)-\nu(0,0)=\pi$ de même: $(-1, 0)\not\in im(t\mapsto e(t, L))\Rightarrow \nu(L,L) - \nu(0, L)=\pi$ donc $2\pi n_C=2\pi$.
		\end{proof}

		\begin{definition}
			Une courbe plane est appelée \textsc{Convexe} si tout ses points sont sur un des cotés de sa tangente. $\Leftrightarrow$ pour chaque $t_C$ $<c(t)-c(t_0)>\geq(\leq) 0,\ \forall t$ avec $n(t_0)\perp T_c(t_0)$.
			% illustration of convexity
		\end{definition}

		\begin{theorem}
			Soit une courbe plane de vitesse 1. Alors:
			\begin{enumerate}
				\item Si $c$ est convexe on a pour sa courbe $\kappa$ on a:
				$$\kappa(t)\geq 0\ \forall t (\mbox{ ou } \kappa(t)\leq 0 \forall t)$$
				\item Si $c$ est fermé simple et si $\kappa(t)\geq 0,\ \forall t $ (ou $\kappa(t)\leq 0, \forall t$) alors $c$ est convexe.
			\end{enumerate}
		\end{theorem}

		\begin{proof}
			\begin{enumerate}
				\item Soit $c$ convexe et supposons que $\expval{c(t)-c(t_0), n(t_0)}\geq 0,\ \forall t$. On developpe $c(t)=c(t_0)+\dot c(t_0)(t-t_0)+\ddot c(t)\frac{(t-t_0)^2}2 + o(|t-t_0|^2)$.
				$0\leq \expval{c(t)-c(t_2), \underbrace{\dot c^\perp(t_0)}_{n(t_0)} }=\underbrace{\expval{\ddot c(t_0, \dot c^\perp(t_0))} }_{\kappa(t_0)}\underbrace{\frac{(t-t_0)^2}2}_{\geq 0}+ o(|t-t_0|^2)$.
				$\Rightarrow \kappa(t_0)\geq 0$ donc $\kappa(t)\geq 0 \forall t\in I$
				\item Supposons que $\kappa(t)\geq 0\forall t$ et que $c$ est fermée simple de période $L$. Si $c$ n'était pas convexe alors il existerait un $t_0$ t.q.:
				$\varphi(t):=\expval{c(t)-c(t_0), \dot c^\perp(t_0)}$, a des valeurs positives et négatives.
				$\varphi$ atteint un maximum eu point $t_2$ et un minimum au point $t_1$ donc $\varphi(t_2)\geq 0$ et $\varphi(t_1)$ et $\varphi(t_1)\leq 0=\varphi(t_0)\leq\varphi(t_2)$ pour un $t_0$. $\dot \varphi (t_1)=0\expval{\dot c(t_1), \dot c^\perp(t_0)}$ donc $\dot c (t_1)=\pm \dot c(t_0)$, $\dot c(t_2)=\pm \dot c(t_0)$. Au moins deux des vecteurs $\dot c(t_0, \dot c (t_1), \dot c (t_2)$ sont donc les mêmes. Soit $s_1, s_2 \in \{t_0, t_1, t_2\}$ t.q. $s_1<s_2$ $\dot c(s_1)=\dot c(s_2)$. On a $\nu(s_2)-\nu(s_1)=2\pi k$ avec $k\in\Z$. $0\leq \kappa (t)\leq\dot\nu(t)$ donc $\nu$ est croissant donc $k\in\mathbb{N}$ de même. $\nu(s_1+L)-\nu(s_2)=2\pi l$ avec $l\in\mathbb{N}$ donc $2\pi n_c=\nu(s_1+L)-\nu(s_1)=2\pi (l+k)=2\pi \mbox{ (Hopf) } \Rightarrow l=0 ou k=0$. Supposons que $k=0$.
				Donc $\nu(t)=cte \forall t\in [s_1, s_2]$ donc $c(s)=c(s_1)+\dot c(s_1)(s-s_1)=c(s_1)+\dot c(t_0)(s-s_1)$ pour $s\in[s_1,s_2]$. donc $\varphi(s)=\expval{c(s)-c(t_0), \dot c^\perp(t_0)}=\expval{c(s_1)-c(t_0), \dot c^\perp(t_0)}=cte$ ce qui n'est pas possible car au moins 2 des points $t_0, t_1, t_2$ sont dans $[s_1, s_2]$.
			\end{enumerate}
		\end{proof}

		\begin{definition}
			Une courbe plane de vitesse 1. On dit que $c$ admet un sommet en $t_0$ si $\dot \kappa(t_0)=0$. (sommet=vertex en anglais)
		\end{definition}

		\begin{examplebox}
			On peut démontrer que l'ellipse à quatres sommets.
		\end{examplebox}

		\begin{remark}
			De manière générale on sait qu'one fonction périodique admet deux points critiques (un maximum et un minimum).
		\end{remark}

		\begin{theorem}{des 4 sommet (four vertex theorem)}
			Soit $c\in C^\infty(\R, \R^2)$ périodique de période $L$ de vitesse 1 et convexe $c$ admet au moins quatre sommets. 
		\end{theorem}

		Pour la preuve on a besoin de 2 lemmes

		\begin{lemme}
			Si l'intersection d'une courbe convexe plane fermée simple avec une droite $G$ contient plus que deux points différents alors $c$ contrent un segment de $G$.
		\end{lemme}

		\begin{remark}
			%img rem 1
		\end{remark}

		\begin{proof}
			Supposons que $c$ est orienté positive convexe = 0 $\kappa(t)\geq 0 \Rightarrow \dot\nu (t)\geq 0$ pour $\nu$ une angle $\dot c(t)=(\cos \nu(t),\sin \nu(t))$ par Hopf: $\nu(L)-\nu(0)=2\pi$ donc $\nu:[0,L]\rightarrow[0,2\pi]+\nu_0$ est croissante et surjective.
		\end{proof}

		Exercice 2
		\begin{enumerate}
			\item Démontrer qu'un segment de droite est la courbe la plus courte (de classe $C^1$) être deux points.
			S: $A,B\in \R^d$, $c:[0,1]\rightarrow \R^d$, $c(0)=A, c(1)=B$. $L(c)=\int_0^1||\dot c(t)||\dd t$.
	
			$c(1)-c(0)=B-A=\int_0^1\dot c(t)\dd t$, $||B-A||=||\int_0^1\dot c(t)\dd t||\leq \int_0^1||\dot c(t) ||\dd t$.
			\item $f(t)=\cos h (t)$ $\gamma (t)=(t, \cos h (t))$. $s(t)=\int_0^t ||\dot \gamma(\tau)|| \dd \tau=\sin h t,\ t\in[0,2]$. On doit trouves $\varphi$ t.q pour $c:=\gamma\circ\varphi$ on a $||\dot c||=1$. $t(s)=arcsin h s,\ s\in[0,\sin h 2]$, $c:(0, \sin h 2)\rightarrow \R^2$. $c(s)=\gamma(arcsinh s)$, $s\in(o, sinh 2)$. $c(s)=(arcsin h s, \sqrt[2]{1+s^2}), s\in(0, sin h2)$.
			\item $\forall t\neq 1:\ \gamma$ est régulier.
		\end{enumerate}

		Exercice 3
		\begin{enumerate}
			\item Démontrer que si $c:\R\rightarrow\R^n$ est une \underline{paramétrisation par longueur d'arc} d'une courbe fermée, alors $c$ est périodique.
	
			Exemple: $t\mapsto (\cos(e^t),\sin(e^t))R=f(t)\ (t\in \R)$. $f$ n'est pas périodique, $f(\R)=S^1$.
	
			Dénoter: si $c$ est une parametrisation t.q. $||\dot c(t)||=1$ alors $c$ est périodique.
			Idée: $d(t+T)=d(t)$ $T$ est période. On definit $\varphi$ en ce fonction de passage. $s(t)=\int_0^t||\dot d(\tau)||\dd{\tau}=\int_0^T||\dot d(\tau)||\dd{\tau}=L+s(t)$.
			$\varphi(u+L)=\varphi(s(t)+L)-\varphi(s(t+T))=t+T=\varphi(u)+T$, $u=s(t)$, $s\circ\varphi(u)=u$, $\varphi$---function inverse function reciproque. $\bar c:=d\circ \varphi$ est une parameter par long d'arc. $\bar c(u+L)=\varphi(s(t)+L)-\varphi(s(t+T))=t+T=\varphi(u)+T$. ($\varphi$ la fonction reciproque de $s$).
		\end{enumerate}

		Homework all the rest.

		\begin{lemme} 
			$c$ une courbe plane fermée simple et convexe. $c$ intersecté une droite un plus de trois points alors $c$ contient un segment de droite.
			\end{lemme} 
			\begin{proof}
				Soit $c;[0,1]\leftarrow  \R$ la courbe.
				Supposons que pour la droite $G=p_0+\R \nu$. $c([0,1])\cap G=\{c(0),c(t_1),c(t_2)\}$. Supposons que $\kapa\geq 0$ donc pour l'angle $\nu$ t.q. $\dot c(t)=(\cos \nu (t),\sin \nu (t))$ an a que $\dot \nu=\kapa\geq 0$ et $\nu(L)=\nu(0)=2\pi$ donc $\nu:[0,L]\leftarrow [0,2\pi]+\nu_0$ est croissante et surjective. Soient $I_j=[t_j,t_{j+1}]$ ($[0,t_1],[t_1,t_2],[t_2,L]$).
				Supposons que $c(I_j)\cap G\neq c(I_j)$. Soit $G_S=G+s\nu^\perp$. Soit $s_1=sup\{s>0;\ G_s\cap c(I_j)\neq 0\}$. Soit $\tau_j$ define par $c(I_j)\cap G_{s_1}=\{c(\tau_j)\}$ donc $\dot c (\tau_j)=\pm\nu$. Donc $\exists \tau_n$ t.q. $0<\tau_1<t_1<\tau_2<t_2<\tau_3<L$ t.q. $c(\tau_n)=\pm \nu\ \forall k$. Soit $\theta_1\in\theta_0+[0,2\pi)$ t.q. $(\cos \theta_n,\sin \theta_n)=\nu$. Supposons que $\theta_2=\theta_1+\pi$ et $(\cos\nu_2,\sin\nu_2)=-\nu$ donc $c(\tau_k)\in\{\theta_1,\theta_2\},\forall k\in\{1,2,3\}$. $t\mapsto\theta(t)$ est croissant donc $\exists j$ t.q. $\theta|_[t_j,t_{j+1}]$ est constant.
			\end{proof}

			\begin{lemme} 
				Soit une courbe plane fermée et sample et convexe. $G$ une droite t.q. $G\cap im(c)=\{p_1,p_2\}$ t.q. $T_{p_1}(c)=T_{p_2}(c)$ colinéaire  $G$  alors $c$ contient un segment de $G$.
				\end{lemme} 
				\begin{proof}
					$G=T_{p_1}(c)$ donc apr convexité la courbe est situé d'un seul coté de $G$ donc supposons:
					$$\expval{c(t)-p_1,\dot c^\perp (t_1)}>0$$
					Soit $G_\eps =G+\eps\dot c^\perp(t_1)$. Pout $\eps$ suffisent petit $G_\eps\cap im(G)=\{q_1,q_2,q_3,q_4\}$ avec $q_j\neq q_k,\ j\neq k,\ q_j\in im(c)$. le résultat suit du lemme précédent.
				\end{proof}

				\begin{theorem}[des 4 sommets]
					soit c une courbe plane, convexe fermé simple alors c admet quatre sommet.
				\end{theorem}
				\begin{proof}
					Supposons que $c$ est paramétrique par longueur d'arc et de période $L$. Pour sa courbure $\kapa$ on sait que $\kapa$ atteint son maximum et son minimum dans $[0,L]$ donc il existent $t_0,t_1\in[0,L)$ t.q. $\dot\kapa(t_j)=0\ j\in\{1,2\}$. Supposons que $t_0=0$. Soit $G=Aff(c(0),c(t_1))$ la droite affine passant parce points. S'il existerait un trois ème point d'intersection de $G$ avec $c$ alors la courbe contiendrait un segment de G (lemme précédant) donc on aurait fini car $\dot\kapa=0$ sur ce segment. Si l'intersection éteint tangentielle en $c(0)$ et $c(t_1)$ alors $c$ on tiendrait un segment de droite parle lemme précédant pour $G=p_0+\R\nu$ on peut donc supposer que:
					\begin{align}		
						\expval{c(t)-c(t_0),\mu^\perp}>0\ & t\in(0,t_1)\\
						\expval{c(t)-c(t_0),\mu^\perp}<0\ & t\in(t_1,L)
					\end{align}
					$\kapa$ est périodique de période $L$ donc $\int\limits_0^L\dot\kapa=0$. Si $\dot\kapa(t)\neq 0\ \forall t\in\{0,t_1\}$. Alors on peut supposer que:
					\begin{align*}
						\dot\kapa (t)>0\ &t\in(t_1,L)\\
						\dot\kapa (t)<0\ &t\in(0,t_1)
					\end{align*}
					=> $\dot\kapa(t)\expval{c(t)-c(0),\nu^\perp}>0,\ t\in(t_1,L)\text{ et }t\in(0,t_1)$ or $\int\dot\kapa(t)(c(t)-c(0))\dd{t}=-\int\limits_0^L\kapa(t)\dot c(t)\dd t$ or on sait que $\dot n(t)=\kapa(t)\dot c(t)$ équation de Frenet $n=\dot c^\perp$.
					\begin{align*}
						\dot T&=\kapa n\\
						\dot N&=-\kapa T
					\end{align*}
					$$\int_0^L\dot\kapa(t)\expval{c(t)-c(0),\nu^\perp}\dd t=\expval{0,\nu^\perp}=0$$
					C'est une contradiction donc il existe un $t_2\in\{0,t_1\}$ t.q. $\dot\kapa(t_2)=0$.

					Supposons que $t_2\in(t_1,L)$. S'il n'y avait pas de quartier sommet. Il existe donc une droite qui sépare les regions $\dot\kapa>0$ et $\dot\kapa<0$. Par le même argument pour ces regions on conclut qu'il existe un 4ème sommet.
				\end{proof}

				\begin{remark}
					Le théorème reste vrai sans l'hypothèse de la convexité.
				\end{remark}

				\section{Inégalité isopérimetrique} % (fold)

				l'aire du cerclée  $rayon\ R=\pi\R^2=A$---area\\
				la $longueur\ 2\pi\R=L$
				$L^2=4\pi^2\R=4\pi A$.
				\begin{theorem}
					Soit $G\subset\R^2$ une region bornée par une courbe fermé simple de longueur L. Alors pour l'aire $A$ de $G$ on a:
					$$4\pi A\leq L^2$$
					et $4\pi A=L^2$ $\Leftrightarrow$la courbe est un cercle.
				\end{theorem}
				\begin{proof}
					Soit $c$ une paramétrisation de la courbe de vitesse 1, de période $L$ orientée positive. Pour déterminer $A$ à partir de $c$ on utilise le théorème de Stoks. Pour $F\in C'(G,\R^2)$ un champs de vecteurs on a:
					$$\int_G \rot F(x,y)\dd{(x,y)}=\int_C \expval{F,\dd{s}}:=\int_0^L \expval{F(c(t)),\dot c(t)}\dd{t}$$
		
					Un F t.q. $\rot F=1$
		
					$F(x,y)=\frac 12 (-y,x)$
		
					$$\rot F(x,y)=\partial_x F2 - \partial_y F_1 = 1$$
		
					donc $\int \rot F=\int_G 1=A=\int\expval{F, \cot c}=\int_0^L(x\dot y-\dot x y)\dd t$ avec $c(t)=(x(t),Y(t))$
		
					On utilise un l'analyse de Fourier. Soit 
					\begin{align*}			
						z:\R &\leftarrow  \C^2\\
						z(t) &:= x(\frac L{2\pi}t)+i y(\frac L{2\pi}t)
					\end{align*}
					alors $x\in C^\infty$ et $z(t+2\pi)=z(t)$ par Fourier on sait $z(t)=\sum\limits_{k\in\Z}c_ke^{ikt}\ \forall t$.
		
					$\dot x(t)=\frac L{2\pi}(\dot x(\frac l{2\pi})+i\dot y(\frac l{2\pi}))$
		
					$|\dot z(t)|^2=\frac{L^2}{(2\pi)^2}(\dot x^2+\dot y^2)(\frac L{2\pi}t)$
		
					$\int_0^{2\pi}|\dot z (t)|^2=\frac{l^2}{2\pi}$
		
					$\dot z(t)=\sum c_k(ik)e^{iht}\ \forall t$
					$|\dot z|^2(t)=\sum_{k,l}(inc_n)(-il\bar c_e)e^{i(k-l)t}$
					$\int_0^{2\pi}|\dot z|^2(t)=\sum_{k,l}\int(...)e^{i(h-l)t}$
					donc:
					$\int_0^{2\pi}|\dot z|^2(t)\dd{t}=\sum_{k\in\Z}k^2|c_n|^2$ donc $\frac{L^2}{2\pi}=\sum k^2|c_n|^2$.
					$Im\dot z\bar z(t)=(\dot yx-x\dot y)(\frac L{2\pi})\frac L{2\pi}$.
		
					$$2A=\frac L{2\pi}\int\limits_0^{2\pi}\Im \dot z\bar z=\sum k|c_k|^2\cdot 2\pi$$
					$$4\pi A=4\pi^2\sum k|c_k|^2$$
					$$L^2=2\pi\cdot \sum k^2 |c_k|^2$$
					or $\sum_{k\in\Z}k|c_k|^2\leq\sum_{k\in\Z}k^2|c_k|^2$ avec égalité $\Leftrightarrow$$c_k=0$ pour $k\not\in\{0,1\}$ donc égalité $\Leftrightarrow$$z(t)=c_0+c_1e^{it}$ $\Leftrightarrow$$t\mapsto (x(t),y(t))$ est un cercle.
		
				\end{proof}
				
\section{Courbes dans $\R^3$} % (fold)

\begin{definition}
	Soit $c\in C^∞(I;\R^3)$ une courbe paramétrie et réguliére.
	\begin{enumerate}
		\item $ν\in C^∞(I;\R^3)$
		$$ν(t):=\frac{\dot c(t)}{||\dot c (t)||}$$
		est appelée \textsc{Champs Tangent}. $c$ est appelé une courbe paramétrie \textsc{Bireguliere} si $\dot v(t) \wedge  \ddot c(t)\neq 0,\ \forall t\in I$. (produit vectoriel).
		Dans ce cas on difinit: $$b(t):=\frac{\dot c(t)\wedge \ddot c(t)}{||\dot c(t)\wedge \ddot c(t)||}$$
		le \textsc{Champs Binormalte} et le plan \textsc{Osculateur}:
		$$\pro_c(t)=\{p\in \R^3:\ \expval{p-c(t),b(t)}=0\}$$ plan affine passant perpendiculaire avec vecteur normale $b(t)$. Le \textsc{Champs Normale} est définie par $n(t):=b(t)\wedge ν(t)$.
		\item Pour une courbe paramétrie biréguliére le \texttt{repére orthomal directe} \\$\{ν(t),n(t),b(t)\}$ est appelé le \textsc{Repére de Frenet} de la courbe $c$ au point $c(t)$.
		$$κ(t):=\frac 1{||\dot c(t)||}\expval{\dot ν(t),n(t)}$$
		est appelée \textsc{Courbure} de coube de $c$ en $t$:
		$$\tilde c(t):=\frac 1{||\dot c(t)||}\expval{\dot n(t), b(t)}$$
		est appelée la \textsc{Torsion} de $c$ en $t$.
	\end{enumerate}
\end{definition}

\begin{remark}
	\begin{enumerate}
		
		\item la biregular assure que le plan osculaleur est bien definie.
		$$\pro_c(t):=c(t)+\vect\{\dot c(t),\ddot c(t)\}$$
		
		\item le vecteur $b(t)\perp\pro_c(t)$.
		\item $n(t)\in \vect\{\dot c(t),\ddot c(t)\}$
		\item $\vect\{\dot c(t), \ddot c(t)\}=\vect\{\nu(t), n(t)\}$
		\item Si $c$ est de vitesse 1 alors $c$ biréguliére $\Leftrightarrow$$||\ddot c (t)||\neq 0,\ \forall t$ car dans ce cas $\expval{\dot c(t),\ddot c(t)}=0$ donc $||\dot c()\wedge \ddot c (t)||=||\dot c(t)||\cdot ||\ddot c(t)||\neq 0$ de plus $κ(t)=||\ddot c(t)||$ (car $κ(t)=\expval{\dot ν,n(t)}=\expval{\dot c(t),\frac{\ddot c(t)}{||\ddot c(t)||}}=||\ddot c(t)||$).
		\item En particulier pour une courbe dans l'espace $κ(t)\geq0\ \forall$
		\item Si $c(I)=imc\subset plan\subset\R^3$ la courbure de $c$ n'est pas même que la courbure definie pour la xstihon $\hat c$ au plan on a $κ=|\hat κ|$.
		\item Ce plan osculateur est indipendant de la parametrisation. $\check c=c\circ φ; \dot\check c=\dot c\circ φ \cdotφ; \ddot\check c=\ddot c\circ φ \dot φ^2+\dot c°φ \ddot φ$.
		($\vect\{\dot\check c(t),\ddot \check c(t)\}=vect\{\dot c(\phi(t)),\ddot c(\phi(t))$).
	\end{enumerate}
\end{remark}

\begin{proposition}
	Equations de Frenet pour une courbe biréguliére.
	\begin{align*}
		\dot ν (t) &= &\frac 1{||\dot c(t)||}κ(t)n(t)\\
		\dot n(t) &= &\frac 1{||\dot c(t)||}(-κ(t)ν(t)+τ(t)b(t))\\
		\dot b(t) &= -&\frac 1{||\dot c(t)||}τ(t)n(t)
	\end{align*}
	Memo $\begin{array}{c}ν\\n\\b\end{array}=\begin{array}{ccc}0&0&0\\0&0&0\\0&0&0 \end{array}\begin{array}{c}\end{array}$.
\end{proposition}
\begin{proof}
	$κ=\frac 1{||\dot c||}\expval{\dot ν,n}=0\ (1)$\\
	$\expval{\dot ν,b}=0$ car $\dot ν\in\vect\{\dot c, \ddot c\}$. $\expval{ν,b}=0$ $\Rightarrow$$\expval{\dot ν,b}+\expval{ν,\dot b}=0$ donc $\dot b\perp ν$.
	$τ=\frac 1{||\dot c||}\expval{\dot n, b}$ $\expval{n,b}=0$ $\expval{\dot n,b}+\expval{n,\dot b}=0$ $\Rightarrow$(3). (2) découle donc de $\expval{\dot n, ν}=-\expval{n,\dot ν}$ car $\expval{ν,n}=0$ $\expval{\dot n, b}$ definition de $\tau$.
\end{proof}

\begin{theorem}
	[foundammentale de la théorie de Frenet]
	Soit $I$ un intervalle et $κ,τ\in C^∞(I,\R)$, $κ(t)\geq 0$. Alors il existe une courbe paramétrie de vitesse 1 $c\in C^∞(I;\R^3)$ tq. sa courbure et sa torsion sont $τ$ et $κ$. Toute autre courbe qui ales mémes propriétés est de la forme: $\hat c=F\circ c$ avec $F(x)=Ax+b$ avec $A\in SO(3)$.
\end{theorem}
\begin{proof}
	Ce systéme d'équations differentielles:
	\begin{align*}
		\dot ν &= κn\\
		\dot n &= -κ\nu + τb\\
		\dot b &= -τn
	\end{align*}
	est lineaire et d'ordre 1. Pour tout systeme orthonue diuct et $\forall t_0\in I:\ \{e_1, e_2, e_3\}$ il existe une solution t.q.
	\begin{align*}
		ν(t_0)&=e_1\\
		n(t_0)&=e_2\\
		b(t_0)&=e_3
	\end{align*}
	on define $c(t_0)+∫_{t_0}^tν$ pour un $c(t_0)\in\R^3$
\end{proof}

\begin{examplebox}[Pour courbure et $\bar c$osion]
	$κ=\frac 1{||\dot c||}\expval{\dot ν, n};\ τ=\frac 1{||\dot c||}\expval{\dot n, b}$.
	Soit
	\begin{align*}
		c(t) &:=(\cos t, \sin t, t),\ t\in\R\\
		\dot c(t)&=(-\sin t, \cos t, 1);\ ||\dot c(t)||^2=2\\
		\ddot c(t)&=(-\cos t, -\sin t, 0)\\
		ν(t)&=\frac 1{\sqrt{2}} (-\sin t, \cos t, q)\\
		b(t)&=\frac{\dot c \wedge  \ddot c}{||\dot c \wedge  \ddot c||}(t)=\frac{(\sin t,-\cos t, 1)}{\sqrt{2}}\\
		n(t)&=-(\cos t, \sin t, 0)\\
		\dot ν(t)&=\frac 1{\sqrt{2}}(-\cos t, -\sin t, 0)\\
		& \expval{\dot ν, n}=\frac 1{\sqrt{2}} \Rightarrow κ=1\\
		\dot n(t)&=-(-\sin t,\cos t, 0)\\
		& \expval{\dot n, b}=\frac1{\sqrt2} \Rightarrow τ=1
	\end{align*}
	Image
\end{examplebox}

\begin{remark}[Theoreme foundamentalle dans le plan]
	Soit $κ\in C^∞(I;\R)$ pour un intervalle $I$. Alors il existe une courbe paramétrie par lagueur d'arc $c$ t.q. sa courbure est $κ$. Toute autre courbe set un $\hat c$ avec les mêmes proprietes est de forme:
	$$\hat c(t)=F\circ c(t+t_0),$$
	pour $t_0\in \R$ et F une isometrie directe $\Leftrightarrow$deplacement.
\end{remark}

Deux résultats sur la géométrie globale des courbes dans l'espace.

\begin{definition}[courbure totale]
	Soit $c\in C^∞(\R;\R^3)$ une courbe paramétrie par longueur d'arc et périodique de période L, $κ\in C^∞(I;\R)$ est sc courbure. Alors $κ(c):=∫_0^Lκ(t)\dd{t}$ est appelé \textsc{courbure totale} de $c$. 	
\end{definition}

\begin{remark}
	Dans le p'au on sait (Hopf) que $κ(c)=±1$ si $c$ est simple.
\end{remark}

On peut dénoutrer
\begin{theorem}[Fenchel]
	Soit $c\in C^∞(\R;\R^3)$ une courbe fermée simple. Alors pour sa courbure totale:
	$$κ(c)\geq2π.$$
	De plus on a $κ(c)=2π$ $\Leftrightarrow$$c$ est un courbe plane et convexe.
\end{theorem}
\begin{proof}
	Sans.
\end{proof}
On peut dénouter
\begin{theorem}[Fary-Tlilnor]
	Soit $c\in C^∞(\R;\R^3)$ une courbe fermée simple. Si $c$ admet un \texttt{noeud} alors pour la courbure totale on a
	$$κ(c)≥4π.$$
\end{theorem}
\begin{remark}
	Si $c$ admet un \texttt{noeud}, c'est à dire on ne peut définir $c$ d'une manière continue en une courbe plane fermée simple.
\end{remark}

\begin{definition}
	Une \textsc{Isotopie} de $\R^3$ est une application.
	$$φ\in C^0([0,1]\times\R^3;\R^3)$$
	t.q. $\forall t\in[0,1]\ φ(t,•)$ est un \texttt{homeomorphism}.
\end{definition}
\begin{definition}
	Deux courbes fermeies simples $c_1, c_2$ sontnt appélé \textsc{Isotope}. S'il existe une isotopie $φ$ t.q.
	$$φ(0,X)=X\ \forall x\in\R^3;\ φ(1, \img(c_0))= \img(c_1).$$
\end{definition}
\begin{definition}\leavevmode
	\begin{itemize}
		\item Un noeud est une class l'equivalence d'une isotopie.
		\item Une courbe fermé simple est \textsc{Sans Noeud}, si elle est isotope à une courbe plane fermée simple.
	\end{itemize}
\end{definition}

\ifcomment
$\Leftrightarrow$ \Leftrightarrow 
\pdv  \pdv
\pd_{x_2} \pd_{x_2}
\mapsto  \mapsto
\rightarrow \rightarrow
\fi

\section{surfaces}

\begin{definition}[Surface régulière]
	Soit $S\subset \R^3$. $S$ est appelé \textsc{Surface Régulière}. Si pour chaque $p\in S$ il existe un ouvert $V\subset \R^3$ t.q. $p\in V$ et s'il existe un ouvert $U\subset\R^2$ et un $F:\underbrace{U}_{\subset \R^2}\rightarrow\R^3$ $C^∞$ t.q. 
	\begin{enumerate}
		\item $F(U)=S\cap V$ et $F:U\rightarrow S\cap V$ est un homéomorphisme (c.a.d. $F|_U$ continue et son inverse $F\dmo|_U$ est continue)
		\item Le Jacobien $Du F$ a $\rank 2$ $\forall u\in U$
	\end{enumerate}
\end{definition}
\begin{remark}
	La matrice jacobienne dans U repère standard:
	$$F(X_1, X_2)=(F_1(X_1, X_2),F_2(X_1, X_2),F_3(X_1, X_2))$$
	$$Du J = 
	\mqty(
	\pd_{x_1} F_1 & \pd_{x_2} F_1\\
	\pd_{x_1} F_2 & \pd_{x_2} F_2\\
	\pd_{x_1} F_3 & \pd_{x_2} F_3
	)$$
	
	$U=(x_1, X_2)$
	$\pd_{x_j} F = \mqty(
	\pd_{x_j} F_1\\
	\pd_{x_j} F_2\\
	\pd_{x_j} F_3
	)$
	
	donc rang $Du F=2$ $\Leftrightarrow$  $\pd_{x_1} F, \pd_{x_2} F$ sont indépendants $\dim \vect\{\pd_{x_1} F, \pd_{x_2} F\}=2$ $\Leftrightarrow$ deux vecteurs tangents à $S$ au point $F(u)$ qui sont indépendant c'est a dire : on peut définir l'espace tangent $\Leftrightarrow$ $||\pd_{x_1} F\wedge\pd_{x_2} F||\neq 0$.
	
	$u_1=(x_1,x_2)$ la ligne $x_2=\ct{const}$ qui passe par $U$. $\R\ni t\mapsto  (x_1,x_2+t)=:c(t)$, $c(0)=u$. $t\mapsto  F(c(t))$ est la courbe correspondante sur $S$.
	
	$\pdv  F(c(t))|_{t=0}=\pdv  F(x_1,x_2+t)|_t=0 = \pd_{x_2} F(x_1, x_2)$
\end{remark}

\begin{definition}
	Pour une surface régulière l'application $F:U\rightarrow S\cap V$ (on encore $(U,F,V)$) \textsc{Paramétrisation Locale} de $S $au point $p$.
	$S\cap V$ est appelé un \textsc{Voisinage de Coordonnées} et les composantes $(u_1,u_2)$ de $u$ t.q. $F(u)=p$ les \textsc{Coordonnées de $p$ par Rapport} à $F$.
\end{definition}
\begin{example}
	Pour $p\in \R3$ et $X_1$, $X_2\in \R^3$ le plan affine $S:=\{X, X=p+u_1X_1+u_2X_2 \}$ est une surface régulière. Car: On peut prendre (pour tout $p\in S$) $V:=\R^3; U:=\R^2$
	$F(u_1, u_2)=p+u_1X_1+u_2X_2$
	
	F es une fonction affine donc F est différentiable. (en tout que fonction de $\R^2\rightarrow\R^3$)
	$F(U)=S=S\cap\R^3 F:U\rightarrow S$ est un homéomorphisme.
\end{example}
\begin{example}[graphe d'une fonction]
	(Une seule paramétrisation!) Soit $U\subset \R^2$ ouvert $f:U\rightarrow\R$ différentiable. $S=\{x=(x_1,x_2,x_3): (x_1,x_2)\in U, x_3=f(x_1, x_2)\}$
	
	On peut prendre de nouveau $V=\R^3$ $U$ (est $U$)
	$F(u_1,u_2):=(u_1, u_2, f(u_1, u_2))$ $F:U\rightarrow\R^3$ est différentiable. $F:U\rightarrow F(U)=S$ est continue $F|_n\dmo$ est la projection orthogonale donc continue. La surface est régulière car 
	$\pd_{u_1}F=(1,0,\pd_{u_1}f(u_1,u_2))$
	$\pd_{u_2}F=(0,1,\pd_{u_2}f(u_1,u_2))$
	$\pd_{u_1}\wedge\pd_{u_1}= (.,.,1)\neq 0$
	
	Addendum: le plan affine est régulier
	$X=p+u_1X_1+u_2X_2$
	$\pd_{u_1}F=X_1$, $\pd_{u_2}F=X_2$
	$\pd_{u_1}F\wedge\pd_{u_2}F=X_1\wedge X_2\neq Si X_1,X_2$ sont indépendantes $\Leftrightarrow$ $\dim \vect\{X_1, X_2\}=2$.
\end{example}

\begin{example}
	$S(=S^2)=\{(x,y,z)\in \R^3; x^2+y^2+z^2=1\}$
	$S$ est une surface régulière?
	Soit $p=(p_1,p_2,p_3)\in S$ t.q. $p_3>0$
	$F(X,Y)=(X,Y, \sqrt{1-x^2-y^2})$ ($x^2+y^2<1$)
	$U:=\{(X,Y);\ x^2+y^2<1\}; V:= \{(x,y,z); z>0\}$
	
	$S\cup V_3$ est le graphe
	de $(X,Y)\mapsto \sqrt{1-x^1-y^2}$ qui est $C^∞$ par l'exemple du graphe on a que $F$ est une paramétrisation en $p $pour chaque $p\in S\cap V_+$
	
	Soit $p\in S$; $p_3<0$
	on choisi $U:=\{(x,y); x^2+y^2<1\}$ $V_-=\{(x,y,z); z<0\}$
	$F_-(x,y):=(x,y,-\sqrt{1-x^2-y^2})$ $(x,y)\in U$
	$V_-=\{(x,y,z); z<0\}$
	parce que $S\cap V_-$ est le graphe de $U\ni (x,y) \mapsto  -\sqrt{1-x^2-y^2}$ qui est différentiel. Par le précédent $(U,F_-, V_-)$ est un voisinage de coordonnées pour chaque point $p\in S$ t.q. $p_3<0$.
	
	$\{p\in S\text{ t.q. } p_2>0 \}$ est le graphe $U\in (x,y)\mapsto \sqrt{1-y^2-z^2}$ donc par le précédent $(U,F_{1_±}, V_{1_±})$ avec $V_{1_±} ={(x,y,z), x>_<0}$ et $F_{1_±}=(y,z,±\sqrt{1-y^2-x^2})$
	De même: $(U, F_{2_±}, V_{2_±})$ avec $V_{2_±}=\{x,y,z x>0\ y<0\}$ $F_{2_±}(X,z)=(x,z,±\sqrt{1-x^2-z^2})$ est un voisinage de coordonnées pour $\{p\in S; p_2>_<0\}$
	
	En résumé: $S^2$ est une surface régulière.
\end{example}

\begin{remark}
	Il nous a falloir 6 paramétrisations pour monter que $S$ est la une surface régulière. On peut faire avec 2 paramétrisations mais pas avec 1.
\end{remark}	
	
\begin{proposition}
	Soit $V_0\subset\R^3$ ouvert $f\in C^∞(V_0; \R)$
	$S:=\{ (x,y,z)\in V_0; f(x,y,z)=0\}$
	Si $\grad f(p)\neq 0 \forall p\in S$ alors $S$ est une surface régulière.
\end{proposition}
\begin{remark}
	\begin{itemize}
		\item $S^2=f\dmo (0)$ pour $f(x,y,z)=x^2+y^2+z^2 -1$
		\item $S$ --- le plan affine $=f\dmo (0)$ de $f(X)=\expval{X-P,n}$ pour un $p\in S$ et $n$ un vecteur normale à $S$.
	\end{itemize}
\end{remark}
\begin{proof}
	Soit $p=(X_0,Y_0,Z_0)$
	$grad f(p)=(\pd_x f(p),\pd_y f(p),\pd_z f(p))\neq (0,0,0)$
	
	Supposons que $\pd_z f(p) \neq 0$. Par le théorème des fonctions implicites il existe un voisinage $V\subset V_b$ de $p$ un voisinage $U\subset \R^2$ de $(X_0,Y_0)$ et une fonction $g\in C^∞ (U,\R)$ t.q. $S\cap V=\{(x,y,g(x,y)); x,y\in U\}$ donc on conclure en utilisant l'exemple du graphe d'une fonction (cad $f(x,y,g(x,y))=0$).
\end{proof}

Attention: la condition $\grad f(p)\neq 0 (p\in S)$ est suffisante mais pas nécessaire. Par exemple $S^2=\tilde f\dmo (0)$
pour $\tilde f(x,y,z)=(x^2+y^2+z^2-1)^2$ $\grad \tilde f(x,y,z)=2(x^2+y^2+z^2-1)2(x,y,z) =0$ si $x^2+y^2+z^2=1$
\begin{example}
	$f(x,y,z)=x^2+y^2-z^2 (x,y,z)\in\R^3$
	$S=f\dmo(0)$
	$\grad f(x,y,z)=2(x,y,-z)=0$ $\Leftrightarrow$ $(x,y,z)=(0,0,0) (0,0,0)\in S$
	
	In faut donc examine $S$ autour (=dans un voisinage) de $(0,0,0)$
	$S=\{ (x,y,z); |z|=\sqrt{x^2+y^2}\}$
	
	$S$ est un double-cône 
	\begin{remark}
		rotation de la courbe $X\mapsto  (X,Z)$ avec $|x|=|y|$ autour de l'axe des $z$
	\end{remark}
	Il ne eut exister de voisinage $V\subset\R^3$ de $(0,0,0)$ et $U\subset \R$ ouvert t.q. $F|_U:U\rightarrow S\cap V$ soit homeomorphe avec $F:U\rightarrow\R^2$ t.q. $Du F$ est de $rang 2$ 
	car pour $p\in S\cup V$ avec $p_3>0$ et $q\in S\cap V$ avec $q_3< 0$ et toute courbe $c:[0,1]\rightarrow S\cap V $avec $c(0)=p$, $c(1)=q$. $\exists t_0$ t.q. $c(t_0)=(0,0,0) $
	or dans $U$ il existent des courbes qui évitent l'origine. C'est à dire $γ\in C^0([0,1], U)$ $γ(0)=F\dmo(q)$ $γ(1)=F\dmo(p)$ $γ(t)\neq F\dmo(0) \forall t\in[0,1]$.
\end{example}