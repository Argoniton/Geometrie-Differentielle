\tableofcontents
% \vspace{2cm} %Add a 2cm space

\begin{abstract}
Plan:
\begin{enumerate}
\item Courbes (plan + espace)
\begin{itemize}
\item étude local
\item étude global
\end{itemize}
\item surfaces dans $\mathbb{R}^3$
\end{enumerate}\end{abstract}
 
\section{Courbes}

\emph{Lesson 1}

% definition of a curve and regular curve
\theoremstyle{definition}
\begin{definition}{Courbe et Courbe Régulière}
\begin{enumerate}
\item Une courbe paramètre dans $\mathbb{R}^3$ est une function $c:I\rightarrow \mathbb{R}^n$ où $I$ est un intervalle de $\mathbb{R}$ et $c$ est lisse ($c$ est infiniment différentielle, $ c \in C^\infty$).
$$I\ni t\mapsto c(t)\in \mathbb{R}^3,$$
$t$ -- paramètre.
\item Une courbe paramétrée est régulièrement si
$$\dot{c}(t) = \frac{\diff}{\diff t}c(t)\neq 0,$$
pour tout $t\in I$.
\end{enumerate}
\end{definition}

Si une courbe est régulière, $c(t)\neq const$. $\dot{c}(t)$ ¡diuge la tangente à la courbe en $c(t)$.

Chaque régulière courbe est tangente à la ligne.

\begin{definition} La trace d'une courbe paramètre $I\ni t \mapsto c(t)\in \mathbb{R}^n$ est image:
$$\{c(t)\ |\ t\in I\} \subset \mathbb{R}^n.$$
\end{definition}

Une cure paramètre est plus une sa trace.

La courbe $$R\ni t \mapsto \left( \begin{array}{c} t^3 \\ 0 \end{array} \right) \in \mathbb{R}^2,$$ 
$trace = \{ \left( \begin{array}{c} x \\ 0 \end{array} \right)\ |\ x\in \mathbb{R} \}$. Et la courbe $$R\ni t \mapsto \left( \begin{array}{c} t \\ 0 \end{array} \right) \in \mathbb{R}^2$$ a la même trace!

$$ \dot{c}_1(t) = \left( \begin{array}{c} 3t^2 \\ 0 \end{array} \right),\ mais\ \dot{c}_2(t) = \left( \begin{array}{c} 1 \\ 0 \end{array} \right).$$

\begin{definition} Si $I\ni t \mapsto c(t)\in \mathbb{R}$ est une courbe paramètre, $J\subset \mathbb{R}$ -- une intervalle et $\varphi: J\rightarrow I$ une function lisse t.q. $\varphi^{-1}: J\rightarrow I$ est également lisse, on disque(?):
$$J\ni t \mapsto c^2(t) = c\circ\varphi(t) \in \mathbb{R}^n,$$
est une reparametrisation  de  $c$.
\end{definition}

Remarque:  
$\dot{\tilde{c}}(t)=\dot{c}\circ\varphi(t)*\dot{\varphi}(t)$. Donc,
$\tilde{c}$ - régulière $\Longleftrightarrow$ $c$ est régulière.
$$\frac{d}{ds}\varphi^{-1}(s)=\frac{1}{\dot{\varphi}\circ\varphi^{-1}(s)}\neq 0$$

$\varphi: J\rightarrow I$ est un diffeomprphisme comme $\dot{\varphi}\neq0$, on a

$$\left\{\begin{array}{rl}
\mbox{soit}\ \dot{\varphi}(t)>0, & \mbox{pour tout } t\in J \\ 
\mbox{soit}\ \dot{\varphi}(t)<0, & \mbox{pour tout } t\in J 
\end{array}\right.,$$

$$\left\{\begin{array}{rl}
\varphi\mbox{ est } \nearrow \\ 
\varphi\mbox{ est } \searrow
\end{array}\right..$$

Si $\varphi$ est $\nearrow$ on dit une la reparametrisation conserve le sens de parcours (l'orientation).
Si $\varphi$ est $\searrow$, la reparam inverse le sens de parours.

\begin{definition}
\begin{enumerate}
\item Une courbe est une classe d'equivalence de courbes parametrie pour la selation:
$$c\sim \tilde{c}\Longleftrightarrow\tilde{c}\mbox{ est une reparemetrisation de }c$$
\item Une courbe on entee est une classe d'equivalence des courbes parametrie pour:
$$c\sim \tilde{c}\Longleftrightarrow\tilde{c}\mbox{ est une reparemetrisation puservantle sense le parours de }c$$
\end{enumerate}
\end{definition}

\begin{definition}
Si $c$ est une courbe paramètre t.q. $|\dot{c}(t)|=1$ pour tout $t\in I$. On dit que c'est paramitee pur sa louger d'arc.
\end{definition}

\begin{proposition}
Si $I\ni t\mapsto c(t)\in \mathbb{R}^n$ est une courbe param reguliere il existe une reparametrisation de $c$ par ca long d'arc:
$$J\ni s\mapsto \tilde{c}(s)=c\circ\varphi(s)\in \mathbb{R}^n$$
$$|\dot{\tilde{c}}(s)|=1 \mbox{ pour tout } s\in J.$$
\end{proposition}

\begin{lemme}
Si $
\begin{array}{rl} 
J_1\ni s \mapsto \tilde{c_1}(s)\mbox{, et }\\ 
J_2\ni s \mapsto \tilde{c_2}(s)
\end{array} $
sont 2 parametr de par long d'arc de la meme courbe $|\dot{c_1}(s)| = 1 = |\dot{c_2}(s)|$.
alors $c_2(s)=c_1(s_0\pm s)$, pour un $s_0\in \mathbb{R}$ et si $c_1$ et $c_2$ ont un pos le meme suis de parcours. Si $c:[a,\ b]\rightarrow \mathbb{R}^n$ est une courbe parametre sa longen est:
$$ L[c] = \int_{a}^{b} |\dot{c}(t)|\, \diff t$$
$$l =\int_{0}^{t} |\dot{c}(u)|\mathop{\dif u} = t$$ % is a curve length from point u(0) to u(t), it equals t. There is just a single param like that up to a constant.
\end{lemme}

\section{Lesson 2} % (fold)
\label{sec:lesson_2}

% section lesson_2 (end)

\begin{definition}
	Une courbe paramétrie $c:R\rightarrow R^d$ est appelie \textsc{Periodique} de periode $p$, si $c(t+p)=c(t),\ \forall t\in R$.
\end{definition}

\begin{definition}
	Une courbe fermee et appeler une \textsc{Courbe Fermee Simple} s'il existe une parametrisation reguliere, periodique de periode $p$ et si: $c_{[0, p)}$ est injectif.
\end{definition}

\begin{definition}
	$c\in C^\infty(I,\ R^2)$ est applee \textsc{Courbe Plane}.
\end{definition}

\begin{definition}
	Soit $c$ une courbe parametree par longueur d'arc (donc une courbe de vitess 1) (donc $||\dot{c}(t)||=1$). Son hamps normale est definie par:
	$$N(T):=\dot{c}^\perp(t),\ t\in I$$
\end{definition}

\begin{remark}
$N(t)=\left(\begin{array}{cr} 0 & -1 \\ 1 & 0\end{array}\right)\dot{c}(t)$. $N$ depend de l'orientation de la courbe.
\end{remark}

Pour chaquet lestome ${\dot{c},\ N(t)}$ est un base otrthonormee direct de $R^2$.

\begin{lemme}
	Soite une courbe vitesse 1, $N$ son alors $\ddot{c}(t)$ est parallier a $N(t)$.
\end{lemme}
\begin{proof}
	Idee $||\dot{c}(t)||=1,\ \forall t \Longleftrightarrow \ddot{c}(t)\perp\dot{c}(t)$.
\end{proof}

\begin{definition}
	Soit $c\in C^\infty(I,\ R^2)$ une courbe plane de vitesse 1, alors $\ddot{c}(t)=\varkappa(t)N(t)$, avec $\varkappa(t):=<\ddot{c}(t),\ N(t)>$.
	$\varkappa(t)$ - scalar.
	
	Alors $\varkappa\in C^\infty (I,\ R)$ et $\varkappa$ est appele la courbe dec?? ($\varkappa(t)$ la courbe du point $c(t)$)
\end{definition}

\begin{theorem}{Formulles de Fenet}
	Soit $c\in C^\infty(I,\ R^2)$ une courbe de vitesse 1.
	
	Soit $\begin{array}{c}T(t):=\dot{c}(t),\\ N(t):=T^\perp (t) \end{array}$, $\{T(t),\ N(t)\}$ - le systeme ortogonale vecteur. Est appeli le \textsc{reppere de Frenet}, ou \textsc{Base de Frenet}. \textsc{Formules de Frenet}:
	$$\begin{array}{c}\dot{T}(t)=\varkappa(t)N(t)\\ \dot{N}(t)=-\varkappa(t)T(T)\end{array}$$
\end{theorem}

\begin{remark}
	$$\frac{\diff}{\diff\, t}\left(
	\begin{array}{c}
		T\\N\end{array}\right) = \left(\begin{array}{rc}0 & \varkappa\\
		-\varkappa & 0
	\end{array}
	\right)=\left(\begin{array}{c}T\\N\end{array}\right)$$
\end{remark}

\begin{lemme}
	Soit $c:C^\infty([a,\ b],\ R^2)$ une courbe plane devitesse, alors il existe $\nu\in C^\infty([a,\ b],\ R)$ t.q. $\dot{c}(t)=(\cos\nu(t),\ \sin\nu(t))$
\end{lemme}

\begin{definition}
	Soit $c\in C^\infty(R,\ R^2)$ une courbe plane, periodique comenode L et de vitesse 1. (en partiquliere reguliere). Soit $\nu\in C^\infty(R,\ R)$.
	Telque $\dot{c}(t)=(\cos \nu (t),\ \sin\nu(t))$ (an dit: une angle de la tangente).
	
	On define le nobn de rotation de la tangente de $c$: $n_c=\frac{1}{2\pi}(\nu(c)-\nu(o))$
\end{definition}
% from $... R^3 ...$ to $... \mathbb{R}^3 ...$ 
% find: \$([^\n\$]*)R([^\n\$]*)\$
% replace: $$1\mathbb{R}$2$
