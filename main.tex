% !TEX encoding = UTF-8 Unicode
\documentclass[french]{article}
 
\usepackage[utf8]{inputenc}
\usepackage[T1]{fontenc}
\usepackage{babel}

\usepackage{amsthm}
\usepackage{thmtools}
\usepackage{mathtools}

 
\theoremstyle{definition}
\newtheorem{definition}{Définition}[section]

\theoremstyle{theorem}
\newtheorem{theorem}{Theorem}[section]

\theoremstyle{proposition}
\newtheorem{proposition}{Proposition}[section]

\theoremstyle{lemme}
\newtheorem{lemme}{Lemme}[section]



\DeclareMathOperator{\tg}{tg}
\DeclareMathOperator{\ctg}{ctg}
\DeclareMathOperator{\dist}{dist}


 
\begin{document}
 
\tableofcontents
 
% \vspace{2cm} %Add a 2cm space
 
\begin{abstract}
Plan:
\begin{enumerate}
\item Courbes (plan + espace)
\begin{itemize}
\item étude local
\item étude global
\end{itemize}
\item surfaces dans $R^3$
\end{enumerate}\end{abstract}
 
\section{Courbes}

\emph{Lesson 1}

% definition of a curve and regular curve
\theoremstyle{definition}
\begin{definition}{Courbe et Courbe Régulière}
\begin{enumerate}
\item Une courbe paramètre dans $R^3$ est une function $c:I\rightarrow R^n$ où $I$ est un intervalle de $R$ et $c$ est lisse = infiniment différentielle ($C^\infty$).
$$I\ni t\mapsto c(t)\in R^3,$$
$t$ -- paramètre.
\item Une courbe paramétrée est régulièrement si
$$\dot{c}(t) = \frac{d}{dt}c(t)\neq 0,$$
pour tout $t\in I$.
\end{enumerate}
\end{definition}

Si une courbe est régulière, $c(t)\neq const$. $\dot{c}(t)$ ¡diuge la tangente à la courbe en $c(t)$.

Chaque régulière courbe est tangente à la ligne.

\begin{definition} La trace d'une courbe paramètre $I\ni t \mapsto c(t)\in R^n$ est image:
$$\{c(t)\ |\ t\in I\} \subset R^n.$$
\end{definition}

Une cure paramètre est plus une sa trace.

La courbe $$R\ni t \mapsto \left( \begin{array}{c} t^3 \\ 0 \end{array} \right) \in R^2,$$ 
$trace = \{ \left( \begin{array}{c} x \\ 0 \end{array} \right)\ |\ x\in R \}$. Et la courbe $$R\ni t \mapsto \left( \begin{array}{c} t \\ 0 \end{array} \right) \in R^2$$ a la même trace!

$$ \dot{c}_1(t) = \left( \begin{array}{c} 3t^2 \\ 0 \end{array} \right),\ mais\ \dot{c}_2(t) = \left( \begin{array}{c} 1 \\ 0 \end{array} \right).$$

\begin{definition} Si $I\ni t \mapsto c(t)\in R$ est une courbe paramètre, $J\subset R$ -- une intervalle et $\varphi: J\rightarrow I$ une function lisse t.q. $\varphi^{-1}: J\rightarrow I$ est également lisse, on disque(?):
$$J\ni t \mapsto c^2(t) = c\circ\varphi(t) \in R^n,$$
est une reparametrisation  de  $c$.
\end{definition}

Remarque:  
$\dot{\tilde{c}}(t)=\dot{c}\circ\varphi(t)*\dot{\varphi}(t)$. Donc,
$\tilde{c}$ - régulière $\Longleftrightarrow$ $c$ est régulière.
$$\frac{d}{ds}\varphi^{-1}(s)=\frac{1}{\dot{\varphi}\circ\varphi^{-1}(s)}\neq 0$$

$\varphi: J\rightarrow I$ est un diffeomprphisme comme $\dot{\varphi}\neq0$, on a

$$\left\{\begin{array}{rl}
\mbox{soit}\ \dot{\varphi}(t)>0, & \mbox{pour tout } t\in J \\ 
\mbox{soit}\ \dot{\varphi}(t)<0, & \mbox{pour tout } t\in J 
\end{array}\right.,$$

$$\left\{\begin{array}{rl}
\varphi\mbox{ est } \nearrow \\ 
\varphi\mbox{ est } \searrow
\end{array}\right..$$

Si $\varphi$ est $\nearrow$ on dit une la reparametrisation conserve le sens de parcours (l'orientation).
Si $\varphi$ est $\searrow$, la reparam inverse le sens de parours.

\begin{definition}
\begin{enumerate}
\item Une courbe est une classe d'equivalence de courbes parametrie pour la selation:
$$c\sim \tilde{c}\Longleftrightarrow\tilde{c}\mbox{ est une reparemetrisation de }c$$
\item Une courbe on entee est une classe d'equivalence des courbes parametrie pour:
$$c\sim \tilde{c}\Longleftrightarrow\tilde{c}\mbox{ est une reparemetrisation puservantle sense le parours de }c$$
\end{enumerate}
\end{definition}

\begin{definition}
Si $c$ est une courbe paramètre t.q. $|\dot{c}(t)|=1$ pour tout $t\in I$. On dit que c'est paramitee pur sa louger d'arc.
\end{definition}

\begin{proposition}
Si $I\ni t\mapsto c(t)\in R^n$ est une courbe param reguliere il existe une reparametrisation de $c$ par ca long d'arc:
$$J\ni s\mapsto \tilde{c}(s)=c\circ\varphi(s)\in R^n$$
$$|\dot{\tilde{c}}(s)|=1 \mbox{ pour tout } s\in J.$$
\end{proposition}

\begin{lemme}
Si $
\begin{array}{rl} 
J_1\ni s \mapsto \tilde{c_1}(s)\mbox{, et }\\ 
J_2\ni s \mapsto \tilde{c_2}(s)
\end{array} $
sont 2 parametr de par long d'arc de la meme courbe $|\dot{c_1}(s)| = 1 = |\dot{c_2}(s)|$.
alors $c_2(s)=c_1(s_0\pm s)$, pour un $s_0\in R$ et si $c_1$ et $c_2$ ont un pos le meme suis de parcours. Si $c:[a,\ b]\rightarrow R^n$ est une courbe parametre sa longen est:
$$ L[c] = \int_{a}^{b} |\dot{c}(t)| dt$$
$$l =\int_{0}^{t} |\dot{c}(u)| du = t$$ % is a curve length from point u(0) to u(t), it equals t. There is just a single param like that up to a constant.
\end{lemme}
\end{document}