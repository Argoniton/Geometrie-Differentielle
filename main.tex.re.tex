\tableofcontents
% \vspace{2cm} %Add a 2cm space

\begin{abstract}
	Plan:
	\begin{enumerate}
		\item Courbes (plan + espace)
		\begin{itemize}
			\item étude local
			\item étude global
		\end{itemize}
		\item surfaces dans $\mathbb{R}^3$
	\end{enumerate}
\end{abstract}
 
	\section{Courbes}

	\emph{Lesson 1}

	% definition of a curve and regular curve
	\theoremstyle{definition}
	\begin{definition}[Courbe et Courbe Régulière]
		\leavevmode
		\begin{enumerate}
			\item Une Courbe Paramètre dans $\mathbb{R}^3$ est une function $c:I\rightarrow \mathbb{R}^n$ où $I$ est un intervalle de $\mathbb{R}$ et $c$ est lisse ($c$ est infiniment différentielle, $ c \in C^\infty$).
			$$I\ni t\mapsto c(t)\in \mathbb{R}^3,$$
			$t$ -- paramètre.
			\item Une courbe paramètre est régulièrement si
			$$\dot{c}(t) = \frac{\diff}{\diff t}c(t)\neq 0,$$
			pour tout $t\in I$.
		\end{enumerate}
	\end{definition}

	Si une courbe est régulière, $c(t)\neq \ct{const}$. $\dot{c}(t)$ désigne la tangente à la courbe en $c(t)$.

	Chaque régulière courbe est tangente à la ligne.

	\begin{definition} La trace d'une courbe paramètre $I\ni t \mapsto c(t)\in \mathbb{R}^n$ est image:
		$$\{c(t)\ |\ t\in I\} \subset \mathbb{R}^n.$$
	\end{definition}

	Une cure paramètre est plus que sa trace.

	La courbe $\R\ni t \mapsto \left( \begin{array}{c} t^3 \\ 0 \end{array} \right) \in \mathbb{R}^2,$
	$\trs = \{ \left( \begin{array}{c} x \\ 0 \end{array} \right)\ |\ x\in \mathbb{R} \}$. Et la courbe $R\ni t \mapsto \left( \begin{array}{c} t \\ 0 \end{array} \right) \in \mathbb{R}^2$ a la même trace!

	$$ \dot{c}_1(t) = \left( \begin{array}{c} 3t^2 \\ 0 \end{array} \right),\ mais\ \dot{c}_2(t) = \left( \begin{array}{c} 1 \\ 0 \end{array} \right).$$

	\begin{definition} Si $I\ni t \mapsto c(t)\in \mathbb{R}$ est une courbe paramètre, $J\subset \mathbb{R}$ -- une intervalle et $\varphi: J\rightarrow I$ une function lisse t.q. $\varphi^{-1}: J\rightarrow I$ est également lisse, on disque(?):
		$$J\ni t \mapsto c^2(t) = c\circ\varphi(t) \in \mathbb{R}^n,$$
		est une reparamétrisation  de  $c$.
	\end{definition}

\begin{remark}  
	$\dot{\tilde{c}}(t)=\dot{c}\circ\varphi(t)*\dot{\varphi}(t)$. Donc,
	$\tilde{c}$ - régulière $\Longleftrightarrow$ $c$ est régulière.
	$$\frac{d}{ds}\varphi^{-1}(s)=\frac{1}{\dot{\varphi}\circ\varphi^{-1}(s)}\neq 0$$

	$\varphi: J\rightarrow I$ est un difféomorphisme comme $\dot{\varphi}\neq0$, on a

	$$\left\{\begin{array}{rl}
	\mbox{soit}\ \dot{\varphi}(t)>0, & \mbox{pour tout } t\in J \\ 
	\mbox{soit}\ \dot{\varphi}(t)<0, & \mbox{pour tout } t\in J 
	\end{array}\right.,$$

	$$\left\{\begin{array}{rl}
	\varphi\mbox{ est } \nearrow \\ 
	\varphi\mbox{ est } \searrow
	\end{array}\right..$$

	Si $\varphi$ est $\nearrow$ on dit une la reparamétrisation conserve le sens de parcours (l'orientation).
	Si $\varphi$ est $\searrow$, la reparam inverse le sens de parours.
\end{remark}

	\begin{definition}
		\leavevmode
		\begin{enumerate}
			\item Une \textdemp{courbe} est une Classe d'Equivalence de Courbes Paramètre pour la relation:
			$$c\sim \tilde{c}\Longleftrightarrow\tilde{c}\mbox{ est une reparamétrisation de }c$$
			\item Une \textdemp{courbe orientée} est une classe d'equivalence des courbes paramètre pour:
			$$c\sim \tilde{c}\Longleftrightarrow\tilde{c}\mbox{ est une reparamétrisation préservante la sens de parcours de }c$$
		\end{enumerate}
	\end{definition}

	\begin{definition}
		Si $c$ est une courbe paramètre t.q. $|\dot{c}(t)|=1$ pour tout $t\in I$. On dit que c'est paramètre pur sa longueur d'arc. 
	\end{definition}

	\begin{proposition}
		Si $I\ni t\mapsto c(t)\in \mathbb{R}^n$ est une courbe paramètre régulière il existe une reparamétrisation de $c$ sa long d'arc:
		$$J\ni s\mapsto \tilde{c}(s)=c\circ\varphi(s)\in \mathbb{R}^n$$
		$$|\dot{\tilde{c}}(s)|=1 \mbox{ pour tout } s\in J.$$
	\end{proposition}

	\begin{lemme}
		Si $\begin{array}{rl} J_1\ni s &\mapsto \tilde{c_1}(s)\\ J_2\ni s &\mapsto \tilde{c_2}(s) \end{array}$ sont 2 paramètre de par long d'arc de la meme courbe $|\dot{c_1}(s)| = 1 = |\dot{c_2}(s)|$.
			alors $c_2(s)=c_1(s_0\pm s)$, pour un $s_0\in \mathbb{R}$ et si $c_1$ et $c_2$ ont un pos le meme suis de parcours. Si $c:[a,\ b]\rightarrow \mathbb{R}^n$ est une courbe paramètre sa longueur est:
			$$ L[c] = \int_{a}^{b} |\dot{c}(t)|\, \dd{}t$$
			$$l =\int_{0}^{t} |\dot{c}(u)|\dd{u} = t$$ % is a curve length from point u(0) to u(t), it equals t. There is just a single param like that up to a constant.
	\end{lemme}


		\begin{definition}
			Une courbe paramétrique $c:R\rightarrow R^d$ est appelée \textsc{Périodique} de période $p$, si $c(t+p)=c(t),\ \forall t\in R$.
		\end{definition}

		\begin{definition}
			Une courbe fermée et appeler une \textsc{Courbe Fermée Simple} s'il existe une parametrisation régulière, périodique de période $p$ et si: $c_{[0, p)}$ est injectif.
		\end{definition}

		\begin{definition}
			$c\in C^\infty(I,\ R^2)$ est appelée \textsc{Courbe Plane}.
		\end{definition}

		\begin{definition}
			Soit $c$ une courbe paramètre par longueur d'arc (donc une courbe de vitesse 1) (donc $||\dot{c}(t)||=1$). Son champs normale est définie par:
			$$N(T):=\dot{c}^\perp(t),\ t\in I$$
		\end{definition}

		\begin{remark}
			$N(t)=\left(\begin{array}{cr} 0 & -1 \\ 1 & 0\end{array}\right)\dot{c}(t)$. $N$ depend de l'orientation de la courbe.
		\end{remark}

		Pour chaque $t$ le système ${\dot{c},\ N(t)}$ est un base orthonormée direct de $R^2$.

		\begin{lemme}
			Soit une courbe vitesse 1, $N$ son champs normals alors $\ddot{c}(t)$ est parallèle a $N(t)$.
		\end{lemme}
		\begin{proof}
			Idee $||\dot{c}(t)||=1,\ \forall t \Longleftrightarrow \ddot{c}(t)\perp\dot{c}(t)$.
		\end{proof}

		\begin{definition}
			Soit $c\in C^\infty(I,\ R^2)$ une courbe plane de vitesse 1, alors $\ddot{c}(t)=κ(t)N(t)$, avec $κ(t):=\expval{\ddot{c}(t),\ N(t)}$.
			$κ(t)$ - scalar.
	
			Alors $κ\in C^\infty (I,\ R)$ et $κ$ est appelé la courbure de $c$ ($κ(t)$ la courbure du point $c(t)$)
		\end{definition}

		\begin{theorem}{Formulas de Frenet}
			Soit $c\in C^\infty(I,\ R^2)$ une courbe de vitesse 1.
	
			Soit $\begin{array}{c}T(t):=\dot{c}(t),\\ N(t):=T^\perp (t) \end{array}$, $\{T(t),\ N(t)\}$ - le systeme ortogonale vecteur. Est appellé le \textsc{Repére de Frenet}, ou \textsc{Base de Frenet}. 
			
			\textsc{Formules de Frenet}:
			$$\begin{array}{rcl}\dot{T}(t)&=&κ(t)N(t)\\ \dot{N}(t)&=&-κ(t)T(T)\end{array}$$
		\end{theorem}

		\begin{remark}
			$$\frac{\diff}{\diff\, t}\left(
			\begin{array}{c}
				T\\N\end{array}\right) = \left(\begin{array}{rc}0 & κ\\
				-κ & 0
			\end{array}
			\right) \left(\begin{array}{c}T\\N\end{array}\right)$$
		\end{remark}

		\begin{lemme}
			Soit $c:C^\infty([a,\ b],\ R^2)$ une courbe plane de vitesse, alors il existe $\nu\in C^\infty([a,\ b],\ R)$ t.q. $\dot{c}(t)=(\cos\nu(t),\ \sin\nu(t))$
		\end{lemme}

		\begin{definition}
			Soit $c\in C^\infty(R,\ R^2)$ une courbe plane, périodique de période L et de vitesse 1. En particulier régulière. Soit $\nu\in C^\infty(R,\ R)$.
			Talque $\dot{c}(t)=(\cos \nu (t),\ \sin\nu(t))$ (an dit: une angle de la tangente).
	
			On define Le Nobre de rotation de la tangente de $c$: $n_c:=\frac{1}{2\pi}(\nu(c)-\nu(o))$
		\end{definition}
		% from $... R^3 ...$ to $... \mathbb{R}^3 ...$ 
		% find: \$([^\n\$]*)R([^\n\$]*)\$
		% replace: $$1\mathbb{R}$2$
\begin{rappel}
$c\in C^\infty(I; \R^2)$ régulière. Alors $\exists\nu\in C^\infty (I; )$ t.q. $\dot{c}(t)=(\cos \nu(t), \sin\nu(t))$. On définie le \textsc{Nombre de Rotation de la Tangente} pour une courbe periodique de période $L$:
		$$n_c:=\frac{1}{2\pi}(\nu (L)-\nu(0))$$
\end{rappel}
	
		\begin{lemme}
			Soient $c_1, c_2\in C^\infty (\R; \R^2)$ deux courbes périodiques de période $L$, paramètre par longueur d'arc $S$: $c_1=c_2\circ\varphi$ avec $\varphi>0 $ alors:
			$$n_{c_1}=n_{c_2}$$
			Si $\dot{\varphi}<0$ alors
			$$n_{c_1}=-n_{c_2}$$
		\end{lemme}
		\begin{remark}
			Le nombre de rotation de la tangente est donc invariant par rapport à une reparamétrisation que preserve l'orientation.
		\end{remark}
		\begin{proof}
			On avait vu que $\varphi(t)=\pm t+t_0$ donc $\dot{\varphi}>0$ $\Rightarrow$  $\varphi (t)=t+t_0$. Soit $\nu_2$ t.q. $\dot{c}_2(t)=(\cos\nu_2(t), \sin\nu_2(t))$ alors pour $\nu_1:=\nu_2\circ\varphi$ on a que $\dot{c}_1(t)=(\cos\nu_1(t), \sin\nu(_1t))$. Soit $\bar{\nu}_1(t):=\nu_1(t+L)$ on a que $\dot{c}(t)=(\cos\bar{\nu}_1(t), \sin\bar{\nu}(_1t))$ car $c_1(t)=c_1(t+L)$.
			\begin{align}
				2\pi(n_{c_2}-n_{c_1}) & = (\nu_2(L)-\nu_2(0))-(\nu_1(L)-\nu_1(0))
				& =(\nu_2(L-t_0)-\nu_2(-t_0)) - (\nu_1(L)-\nu_1(0))
				& = ...
				=0
			\end{align}
		\end{proof}

		\begin{theorem}
			Sait $c$ une courbe plane périodique de période $L$ et paramètre par longueur d'arc. Soit $\kappa$ la courbure de $c$ alors
			$$n_c=\frac{1}{2\pi}\int_0^L\kappa(t)\dd{t}$$
		\end{theorem}
		
		\begin{remark}
			En particulier $\int\limits_0^L\kappa(t)\dd{t}\in 2\pi\mathbb{Z}$
		\end{remark}
		\begin{proof}
			Soit $\nu\in C^\infty(\R, \R)$ une fonction angle pour la tangente, c.à.d. $\dot{c}(t)=(\cos \nu(t), \sin\nu(t))$. $\ddot{c}(t)=\kappa(t)\dot{c}^\perp (t)$ donc $\kappa(t)=\expval{\ddot{c}(t), \dot{c}^\perp (t)}$ ou $\ddot{c}(t)=\dot{\nu}(t)(-\sin\nu(t), \cos\nu(t))$ et $\dot{c}^\perp(t)=(-\sin\nu(t),\cos\nu(t))$
			donc $<\ddot{c}(t), \dot{c}^\perp(t)>=\dot{\nu}(t)=\kappa(t)$ ou
			$$n_c = \frac{1}{2\pi}(\nu(L)-\nu(0))=\frac{1}{2\pi}\int_0^L \dot{\nu}(t)\dd{t}=\frac{1}{2\pi}\int_0^L\kappa(t)\dd{t}.$$
		\end{proof}

		\begin{theorem}[Hopf. Turning tangent theorem]
			Une courbe plane fermée simple a un nombre de rotation (de la tangente) 1 ou $-1$.
		\end{theorem}

		Nombre de rotation $n=\frac{1}{2\pi}\int\limits_0^L\kappa(t)\dd{t}=\frac{1}{2\pi}(\nu(L)-\nu(0))$. $c(t+l)=c(t)$ $c(t)=(\cos \nu (t), \sin \nu(t)),\ \dot\nu=\kappa$
		\begin{remark}
			On avait inclu dans la défini de fermée simple qu'il n'ya pas de point singulier.
		\end{remark}
	
		Pour la preuve on aura besoin du lemme de recouvrement.

		\begin{definition}
			Sait $X\subset \R^d$ et $x_0\in X$ On dit que $X$ est \textsc{Étoile} par rapport à $x_0$, ($X$ is star shaped). Si pour chaque $x\in X$ le segment de droite entre $x_0$ et $x$ est contenu dans $X$. C'est dire $\forall x$ $\{x_0{1-t}+xt, t\in[0,1]\}\subset X$
		\end{definition}

		\begin{lemme}{De Recouvrement}
			Soit $X\subset \R^d$ étoilé par rapport à $x_0$ et soit
			$$e: X\rightarrow S^1=\{(x,y)\in\R^2, x^2+y^2=1\} \text{---une application continue}$$
	
			Alors in existe une application \underline{continue} $\nu: X\rightarrow \R$ t.q. $e(x)=(\cos\nu(x), \sin\nu(x))$. $\nu$ est unique sous la condition $\nu(x_0)=\nu_0$.
		\end{lemme}

		\begin{proof}
			\underline{Cas} ou $e: X\rightarrow S^1$ n'est pas surjective. Supposons qu'il existe $\varphi_0\in \R$ t.q. $(\cos\varphi_0, \sin\varphi_0)\notin e(X)$. $e(X)=\{z; z=e(x), x\in X\}$.
			La fonction $\psi:(\varphi_0, \varphi_0+2\pi)\rightarrow S^1\\\{(\cos\varphi_0, \sin\varphi_2\}$ est un homéomorphisme.
			On $\nu=\psi^{-1}\circ e$ donc $\nu$ est continue.
	
			\underline{Cas} $e(X)=S^1$. Dans le cas $d=1$, $X=[0,1]$, $x_0=0$ on a démontré le théorème ($e=\dot{c}$ dériver d'une courbe) %img 4planes
	
			\underline{Cas} $d>1$. Soit $x\in X$. On defini $e_x:[0,1]\rightarrow S^1$, $e(x)(t)=e(tx+(1-t)x_0)$. %img a star
			On sait qu'il existe $\nu_x:[0,1]\rightarrow \R$ continue t.q. $e_x(t)=(\cos\nu_x(t),\sin\nu_x(t))$ de $\nu_x(t)=\nu(tx+(1-t)x_0)$ donc $\nu(x)=\nu_x(1)$ donc
			$e(x)=e_x(1)=(\cos\nu_x(1), \sin\nu_x(1))$ is est e a de monte que $\nu_x(1)$ est continue en $e$.
	
			Soit $\eps>0$ et $0=t_0<t_1<t_2<...<t_n=1$ une partition t.q. $e_x|_{[t_j, t_{j+1}]}\subset U_h,\ H\in\{1,2,3,4\}$. Soit $y$ t.q. $\norm{e_x(t)-e_y(t)}<\eps,\ \forall t\in [0,1]$. Si $\eps$ est suffisent petit. $e_y|_{[t_j, t_{j+1})}\subset U_h$. Par example dans le cas $h=4$ on aura
			\begin{align}
				\nu_x(t) &=\arctan\left(\frac{e_x^2(t)}{e_x^1(t)}\right)
				\nu_y(t) &=\arctan\left(\frac{e_y^2(t)}{e_y^1(t)}\right)		
			\end{align}
			$e=(e^1, e^2)$
		\end{proof}

		\begin{proof}{du théorème de Hopf}
			%%img kidney
			Soit $c$ une une paramétrisation de vitesse 1 de période $L$. Sait $x_0:=\max\{c^1(t); t\in [0, l]\}$. Soit $p=\{(z_1, z_2); z_1=x_0\}\cap C(\R)$
			Soit la paramétrisation t.q. $c(0)=p$. $G=p+\R(1,0)$. $C(\R)\cap G$ est à gauche de $p$.
			Soit $X=\{(t_1, t_2): 0\leq t_1\leq t_2 \leq L\}$ %img trin
			$X$ est étoilé par rapport à $(0,0)$. On considère $c:X\rightarrow S^1$
			Formula after an image.
			$$c(t_1,t_2)=\left\{ \begin{array}{cr}\frac{c(t_1)-c(t_1)}{||c(t_1)-c(t_1)||} & t_2>t_1 \\ \dot c(t) & t_2=t_1=t \\ -\dot c(0) & (t_1, t_2)=(0,L)\end{array}\right.$$
	
			Alors $e\in C^0(x, S^1)$, en effet $c\in C^\infty.$ $c(t_2)=c(t_1)+\dot c(t_1)(t_2-t_1)+o(|t_2-t_1|)$
	
			$$\frac{c(t_1)-c(t_1)}{||c(t_1)-c(t_1)||}=\frac{(t_2-t_1)(\dot c(t_1)-o(1))}{||(t_2-t_1)(\dot c(t_1)-o(1))||}\to \frac{\dot c(t_1)}{||\dot c(t_1)||}=\dot c(t_1)$$
			$$t_2\to t_1$$
			$$\frac{c(L-\eps)-c(0)}{||c(L-\eps)-c(0)||}=\frac{c(-\eps)-c(0)}{c(-\eps)-c(0)}=\frac{-\eps(\dot c(0)+o(1))}{||-\eps(\dot c(0)+o(1))||}\to -\dot c(0)$$
			$$\eps\to(down) +0+$$
	
			De plus X est étoilée par rapport à $(0,0)$. Donc il exist $\nu\in C^0(X)$ t.q. $e(t_1, t_2)=(\cos \nu(t_1,t_2), \sin \nu(t_1,t_2))$. Pour de nombre de rotation de ( la tangente de) on a:
			$$2\pi n_c=\nu(L,L)-\nu(0,0)=\nu(L,L) - \nu(0,L)+\nu(0,L)-\nu(0,0)$$
	
			%img nut.ai
	
			(droite $\perp$ à $\dot c(0)$) $x_0=\max \{ c^{(1)},\ t\in [0, L]\}$ $(1,0)\not\in im([0,1]\ni t\mapsto e(0,t))$ car en $c(0), t\mapsto x(t)$ est maximal, donc $im([0,1]\ni t\mapsto \nu(0,t))\subset (0,2\pi)+2\pi k$ (car facile du lemme du recouvrement).
	
			$e(0,L)=-\dot c(0)=(0,-1)$ donc $\nu (0,L)=\frac{3\pi}{2}+2\pi k$ de $\nu(0,0)=\frac{\pi}2+2\pi k $ donc $\nu(0,L)-\nu(0,0)=\pi$ de même: $(-1, 0)\not\in im(t\mapsto e(t, L))\Rightarrow \nu(L,L) - \nu(0, L)=\pi$ donc $2\pi n_C=2\pi$.
		\end{proof}

		\begin{definition}
			Une courbe plane est appelée \textsc{Convexe} si tout ses points sont sur un des cotés de sa tangente. $\Leftrightarrow$ pour chaque $t_C$ $<c(t)-c(t_0)>\geq(\leq) 0,\ \forall t$ avec $n(t_0)\perp T_c(t_0)$.
			% illustration of convexity
		\end{definition}

		\begin{theorem}
			Soit une courbe plane de vitesse 1. Alors:
			\begin{enumerate}
				\item Si $c$ est convexe on a pour sa courbe $\kappa$ on a:
				$$\kappa(t)\geq 0\ \forall t (\mbox{ ou } \kappa(t)\leq 0 \forall t)$$
				\item Si $c$ est fermé simple et si $\kappa(t)\geq 0,\ \forall t $ (ou $\kappa(t)\leq 0, \forall t$) alors $c$ est convexe.
			\end{enumerate}
		\end{theorem}

		\begin{proof}
			\begin{enumerate}
				\item Soit $c$ convexe et supposons que $\expval{c(t)-c(t_0), n(t_0)}\geq 0,\ \forall t$. On developpe $c(t)=c(t_0)+\dot c(t_0)(t-t_0)+\ddot c(t)\frac{(t-t_0)^2}2 + o(|t-t_0|^2)$.
				$0\leq \expval{c(t)-c(t_2), \underbrace{\dot c^\perp(t_0)}_{n(t_0)} }=\underbrace{\expval{\ddot c(t_0, \dot c^\perp(t_0))} }_{\kappa(t_0)}\underbrace{\frac{(t-t_0)^2}2}_{\geq 0}+ o(|t-t_0|^2)$.
				$\Rightarrow \kappa(t_0)\geq 0$ donc $\kappa(t)\geq 0 \forall t\in I$
				\item Supposons que $\kappa(t)\geq 0\forall t$ et que $c$ est fermée simple de période $L$. Si $c$ n'était pas convexe alors il existerait un $t_0$ t.q.:
				$\varphi(t):=\expval{c(t)-c(t_0), \dot c^\perp(t_0)}$, a des valeurs positives et négatives.
				$\varphi$ atteint un maximum eu point $t_2$ et un minimum au point $t_1$ donc $\varphi(t_2)\geq 0$ et $\varphi(t_1)$ et $\varphi(t_1)\leq 0=\varphi(t_0)\leq\varphi(t_2)$ pour un $t_0$. $\dot \varphi (t_1)=0\expval{\dot c(t_1), \dot c^\perp(t_0)}$ donc $\dot c (t_1)=\pm \dot c(t_0)$, $\dot c(t_2)=\pm \dot c(t_0)$. Au moins deux des vecteurs $\dot c(t_0, \dot c (t_1), \dot c (t_2)$ sont donc les mêmes. Soit $s_1, s_2 \in \{t_0, t_1, t_2\}$ t.q. $s_1<s_2$ $\dot c(s_1)=\dot c(s_2)$. On a $\nu(s_2)-\nu(s_1)=2\pi k$ avec $k\in\Z$. $0\leq \kappa (t)\leq\dot\nu(t)$ donc $\nu$ est croissant donc $k\in\mathbb{N}$ de même. $\nu(s_1+L)-\nu(s_2)=2\pi l$ avec $l\in\mathbb{N}$ donc $2\pi n_c=\nu(s_1+L)-\nu(s_1)=2\pi (l+k)=2\pi \mbox{ (Hopf) } \Rightarrow l=0 ou k=0$. Supposons que $k=0$.
				Donc $\nu(t)=cte \forall t\in [s_1, s_2]$ donc $c(s)=c(s_1)+\dot c(s_1)(s-s_1)=c(s_1)+\dot c(t_0)(s-s_1)$ pour $s\in[s_1,s_2]$. donc $\varphi(s)=\expval{c(s)-c(t_0), \dot c^\perp(t_0)}=\expval{c(s_1)-c(t_0), \dot c^\perp(t_0)}=cte$ ce qui n'est pas possible car au moins 2 des points $t_0, t_1, t_2$ sont dans $[s_1, s_2]$.
			\end{enumerate}
		\end{proof}

		\begin{definition}
			Une courbe plane de vitesse 1. On dit que $c$ admet un sommet en $t_0$ si $\dot \kappa(t_0)=0$. (sommet=vertex en anglais)
		\end{definition}

		\begin{examplebox}
			On peut démontrer que l'ellipse à quatres sommets.
		\end{examplebox}

		\begin{remark}
			De manière générale on sait qu'one fonction périodique admet deux points critiques (un maximum et un minimum).
		\end{remark}

		\begin{theorem}{des 4 sommet (four vertex theorem)}
			Soit $c\in C^\infty(\R, \R^2)$ périodique de période $L$ de vitesse 1 et convexe $c$ admet au moins quatre sommets. 
		\end{theorem}

		Pour la preuve on a besoin de 2 lemmes

		\begin{lemme}
			Si l'intersection d'une courbe convexe plane fermée simple avec une droite $G$ contient plus que deux points différents alors $c$ contrent un segment de $G$.
		\end{lemme}

		\begin{remark}
			%img rem 1
		\end{remark}

		\begin{proof}
			Supposons que $c$ est orienté positive convexe = 0 $\kappa(t)\geq 0 \Rightarrow \dot\nu (t)\geq 0$ pour $\nu$ une angle $\dot c(t)=(\cos \nu(t),\sin \nu(t))$ par Hopf: $\nu(L)-\nu(0)=2\pi$ donc $\nu:[0,L]\rightarrow[0,2\pi]+\nu_0$ est croissante et surjective.
		\end{proof}

		Exercice 2
		\begin{enumerate}
			\item Démontrer qu'un segment de droite est la courbe la plus courte (de classe $C^1$) être deux points.
			S: $A,B\in \R^d$, $c:[0,1]\rightarrow \R^d$, $c(0)=A, c(1)=B$. $L(c)=\int_0^1||\dot c(t)||\dd t$.
	
			$c(1)-c(0)=B-A=\int_0^1\dot c(t)\dd t$, $||B-A||=||\int_0^1\dot c(t)\dd t||\leq \int_0^1||\dot c(t) ||\dd t$.
			\item $f(t)=\cos h (t)$ $\gamma (t)=(t, \cos h (t))$. $s(t)=\int_0^t ||\dot \gamma(\tau)|| \dd \tau=\sin h t,\ t\in[0,2]$. On doit trouves $\varphi$ t.q pour $c:=\gamma\circ\varphi$ on a $||\dot c||=1$. $t(s)=arcsin h s,\ s\in[0,\sin h 2]$, $c:(0, \sin h 2)\rightarrow \R^2$. $c(s)=\gamma(arcsinh s)$, $s\in(o, sinh 2)$. $c(s)=(arcsin h s, \sqrt[2]{1+s^2}), s\in(0, sin h2)$.
			\item $\forall t\neq 1:\ \gamma$ est régulier.
		\end{enumerate}

		Exercice 3
		\begin{enumerate}
			\item Démontrer que si $c:\R\rightarrow\R^n$ est une \underline{paramétrisation par longueur d'arc} d'une courbe fermée, alors $c$ est périodique.
	
			Exemple: $t\mapsto (\cos(e^t),\sin(e^t))R=f(t)\ (t\in \R)$. $f$ n'est pas périodique, $f(\R)=S^1$.
	
			Dénoter: si $c$ est une parametrisation t.q. $||\dot c(t)||=1$ alors $c$ est périodique.
			Idée: $d(t+T)=d(t)$ $T$ est période. On definit $\varphi$ en ce fonction de passage. $s(t)=\int_0^t||\dot d(\tau)||\dd{\tau}=\int_0^T||\dot d(\tau)||\dd{\tau}=L+s(t)$.
			$\varphi(u+L)=\varphi(s(t)+L)-\varphi(s(t+T))=t+T=\varphi(u)+T$, $u=s(t)$, $s\circ\varphi(u)=u$, $\varphi$---function inverse function reciproque. $\bar c:=d\circ \varphi$ est une parameter par long d'arc. $\bar c(u+L)=\varphi(s(t)+L)-\varphi(s(t+T))=t+T=\varphi(u)+T$. ($\varphi$ la fonction reciproque de $s$).
		\end{enumerate}

		Homework all the rest.

		\begin{lemme} 
			$c$ une courbe plane fermée simple et convexe. $c$ intersecté une droite un plus de trois points alors $c$ contient un segment de droite.
			\end{lemme} 
			\begin{proof}
				Soit $c;[0,1]\leftarrow  \R$ la courbe.
				Supposons que pour la droite $G=p_0+\R \nu$. $c([0,1])\cap G=\{c(0),c(t_1),c(t_2)\}$. Supposons que $\kapa\geq 0$ donc pour l'angle $\nu$ t.q. $\dot c(t)=(\cos \nu (t),\sin \nu (t))$ an a que $\dot \nu=\kapa\geq 0$ et $\nu(L)=\nu(0)=2\pi$ donc $\nu:[0,L]\leftarrow [0,2\pi]+\nu_0$ est croissante et surjective. Soient $I_j=[t_j,t_{j+1}]$ ($[0,t_1],[t_1,t_2],[t_2,L]$).
				Supposons que $c(I_j)\cap G\neq c(I_j)$. Soit $G_S=G+s\nu^\perp$. Soit $s_1=sup\{s>0;\ G_s\cap c(I_j)\neq 0\}$. Soit $\tau_j$ define par $c(I_j)\cap G_{s_1}=\{c(\tau_j)\}$ donc $\dot c (\tau_j)=\pm\nu$. Donc $\exists \tau_n$ t.q. $0<\tau_1<t_1<\tau_2<t_2<\tau_3<L$ t.q. $c(\tau_n)=\pm \nu\ \forall k$. Soit $\theta_1\in\theta_0+[0,2\pi)$ t.q. $(\cos \theta_n,\sin \theta_n)=\nu$. Supposons que $\theta_2=\theta_1+\pi$ et $(\cos\nu_2,\sin\nu_2)=-\nu$ donc $c(\tau_k)\in\{\theta_1,\theta_2\},\forall k\in\{1,2,3\}$. $t\mapsto\theta(t)$ est croissant donc $\exists j$ t.q. $\theta|_[t_j,t_{j+1}]$ est constant.
			\end{proof}

			\begin{lemme} 
				Soit une courbe plane fermée et sample et convexe. $G$ une droite t.q. $G\cap im(c)=\{p_1,p_2\}$ t.q. $T_{p_1}(c)=T_{p_2}(c)$ colinéaire  $G$  alors $c$ contient un segment de $G$.
				\end{lemme} 
				\begin{proof}
					$G=T_{p_1}(c)$ donc apr convexité la courbe est situé d'un seul coté de $G$ donc supposons:
					$$\expval{c(t)-p_1,\dot c^\perp (t_1)}>0$$
					Soit $G_\eps =G+\eps\dot c^\perp(t_1)$. Pout $\eps$ suffisent petit $G_\eps\cap im(G)=\{q_1,q_2,q_3,q_4\}$ avec $q_j\neq q_k,\ j\neq k,\ q_j\in im(c)$. le résultat suit du lemme précédent.
				\end{proof}

				\begin{theorem}[des 4 sommets]
					soit c une courbe plane, convexe fermé simple alors c admet quatre sommet.
				\end{theorem}
				\begin{proof}
					Supposons que $c$ est paramétrique par longueur d'arc et de période $L$. Pour sa courbure $\kapa$ on sait que $\kapa$ atteint son maximum et son minimum dans $[0,L]$ donc il existent $t_0,t_1\in[0,L)$ t.q. $\dot\kapa(t_j)=0\ j\in\{1,2\}$. Supposons que $t_0=0$. Soit $G=Aff(c(0),c(t_1))$ la droite affine passant parce points. S'il existerait un trois ème point d'intersection de $G$ avec $c$ alors la courbe contiendrait un segment de G (lemme précédant) donc on aurait fini car $\dot\kapa=0$ sur ce segment. Si l'intersection éteint tangentielle en $c(0)$ et $c(t_1)$ alors $c$ on tiendrait un segment de droite parle lemme précédant pour $G=p_0+\R\nu$ on peut donc supposer que:
					\begin{align}		
						\expval{c(t)-c(t_0),\mu^\perp}>0\ & t\in(0,t_1)\\
						\expval{c(t)-c(t_0),\mu^\perp}<0\ & t\in(t_1,L)
					\end{align}
					$\kapa$ est périodique de période $L$ donc $\int\limits_0^L\dot\kapa=0$. Si $\dot\kapa(t)\neq 0\ \forall t\in\{0,t_1\}$. Alors on peut supposer que:
					\begin{align*}
						\dot\kapa (t)>0\ &t\in(t_1,L)\\
						\dot\kapa (t)<0\ &t\in(0,t_1)
					\end{align*}
					$\Rightarrow$  $\dot\kapa(t)\expval{c(t)-c(0),\nu^\perp}>0,\ t\in(t_1,L)\text{ et }t\in(0,t_1)$ or $\int\dot\kapa(t)(c(t)-c(0))\dd{t}=-\int\limits_0^L\kapa(t)\dot c(t)\dd t$ or on sait que $\dot n(t)=\kapa(t)\dot c(t)$ équation de Frenet $n=\dot c^\perp$.
					\begin{align*}
						\dot T&=\kapa n\\
						\dot N&=-\kapa T
					\end{align*}
					$$\int_0^L\dot\kapa(t)\expval{c(t)-c(0),\nu^\perp}\dd t=\expval{0,\nu^\perp}=0$$
					C'est une contradiction donc il existe un $t_2\in\{0,t_1\}$ t.q. $\dot\kapa(t_2)=0$.

					Supposons que $t_2\in(t_1,L)$. S'il n'y avait pas de quartier sommet. Il existe donc une droite qui sépare les regions $\dot\kapa>0$ et $\dot\kapa<0$. Par le même argument pour ces regions on conclut qu'il existe un 4ème sommet.
				\end{proof}

				\begin{remark}
					Le théorème reste vrai sans l'hypothèse de la convexité.
				\end{remark}

				\section{Inégalité isopérimetrique} % (fold)

				l'aire du cerclée  $rayon\ R=\pi\R^2=A$---area\\
				la $longueur\ 2\pi\R=L$
				$L^2=4\pi^2\R=4\pi A$.
				\begin{theorem}
					Soit $G\subset\R^2$ une region bornée par une courbe fermé simple de longueur L. Alors pour l'aire $A$ de $G$ on a:
					$$4\pi A\leq L^2$$
					et $4\pi A=L^2$ $\Leftrightarrow$la courbe est un cercle.
				\end{theorem}
				\begin{proof}
					Soit $c$ une paramétrisation de la courbe de vitesse 1, de période $L$ orientée positive. Pour déterminer $A$ à partir de $c$ on utilise le théorème de Stoks. Pour $F\in C'(G,\R^2)$ un champs de vecteurs on a:
					$$\int_G \rot F(x,y)\dd{(x,y)}=\int_C \expval{F,\dd{s}}:=\int_0^L \expval{F(c(t)),\dot c(t)}\dd{t}$$
		
					Un F t.q. $\rot F=1$
		
					$F(x,y)=\frac 12 (-y,x)$
		
					$$\rot F(x,y)=\partial_x F2 - \partial_y F_1 = 1$$
		
					donc $\int \rot F=\int_G 1=A=\int\expval{F, \cot c}=\int_0^L(x\dot y-\dot x y)\dd t$ avec $c(t)=(x(t),Y(t))$
		
					On utilise un l'analyse de Fourier. Soit 
					\begin{align*}			
						z:\R &\leftarrow  \C^2\\
						z(t) &:= x(\frac L{2\pi}t)+i y(\frac L{2\pi}t)
					\end{align*}
					alors $x\in C^\infty$ et $z(t+2\pi)=z(t)$ par Fourier on sait $z(t)=\sum\limits_{k\in\Z}c_ke^{ikt}\ \forall t$.
		
					$\dot x(t)=\frac L{2\pi}(\dot x(\frac l{2\pi})+i\dot y(\frac l{2\pi}))$
		
					$|\dot z(t)|^2=\frac{L^2}{(2\pi)^2}(\dot x^2+\dot y^2)(\frac L{2\pi}t)$
		
					$\int_0^{2\pi}|\dot z (t)|^2=\frac{l^2}{2\pi}$
		
					$\dot z(t)=\sum c_k(ik)e^{iht}\ \forall t$
					$|\dot z|^2(t)=\sum_{k,l}(inc_n)(-il\bar c_e)e^{i(k-l)t}$
					$\int_0^{2\pi}|\dot z|^2(t)=\sum_{k,l}\int(...)e^{i(h-l)t}$
					donc:
					$\int_0^{2\pi}|\dot z|^2(t)\dd{t}=\sum_{k\in\Z}k^2|c_n|^2$ donc $\frac{L^2}{2\pi}=\sum k^2|c_n|^2$.
					$Im\dot z\bar z(t)=(\dot yx-x\dot y)(\frac L{2\pi})\frac L{2\pi}$.
		
					$$2A=\frac L{2\pi}\int\limits_0^{2\pi}\Im \dot z\bar z=\sum k|c_k|^2\cdot 2\pi$$
					$$4\pi A=4\pi^2\sum k|c_k|^2$$
					$$L^2=2\pi\cdot \sum k^2 |c_k|^2$$
					or $\sum_{k\in\Z}k|c_k|^2\leq\sum_{k\in\Z}k^2|c_k|^2$ avec égalité $\Leftrightarrow$$c_k=0$ pour $k\not\in\{0,1\}$ donc égalité $\Leftrightarrow$$z(t)=c_0+c_1e^{it}$ $\Leftrightarrow$$t\mapsto (x(t),y(t))$ est un cercle.
		
				\end{proof}
				
\section{Courbes dans $\R^3$} % (fold)

\begin{definition}
	Soit $c\in C^∞(I;\R^3)$ une courbe paramétrie et réguliére.
	\begin{enumerate}
		\item $ν\in C^∞(I;\R^3)$
		$$ν(t):=\frac{\dot c(t)}{||\dot c (t)||}$$
		est appelée \textsc{Champs Tangent}. $c$ est appelé une courbe paramétrie \textsc{Birégulière} si $\dot v(t) \wedge  \ddot c(t)\neq 0,\ \forall t\in I$. (produit vectoriel).
		Dans ce cas on difinit: $$b(t):=\frac{\dot c(t)\wedge \ddot c(t)}{||\dot c(t)\wedge \ddot c(t)||}$$
		le \textsc{Champs Binormalte} et le plan \textsc{Osculateur}:
		$$\pro_c(t)=\{p\in \R^3:\ \expval{p-c(t),b(t)}=0\}$$ plan affine passant perpendiculaire avec vecteur normale $b(t)$. Le \textsc{Champs Normale} est définie par $n(t):=b(t)\wedge ν(t)$.
		\item Pour une courbe paramétrie biréguliére le \texttt{repére orthomal directe} \\$\{ν(t),n(t),b(t)\}$ est appelé le \textsc{Repére de Frenet} de la courbe $c$ au point $c(t)$.
		$$κ(t):=\frac 1{||\dot c(t)||}\expval{\dot ν(t),n(t)}$$
		est appelée \textsc{Courbure} de coube de $c$ en $t$:
		$$\tilde c(t):=\frac 1{||\dot c(t)||}\expval{\dot n(t), b(t)}$$
		est appelée la \textsc{Torsion} de $c$ en $t$.
	\end{enumerate}
\end{definition}

\begin{remark}
	\begin{enumerate}
		
		\item la biregular assure que le plan osculaleur est bien definie.
		$$\pro_c(t):=c(t)+\vect\{\dot c(t),\ddot c(t)\}$$
		
		\item le vecteur $b(t)\perp\pro_c(t)$.
		\item $n(t)\in \vect\{\dot c(t),\ddot c(t)\}$
		\item $\vect\{\dot c(t), \ddot c(t)\}=\vect\{\nu(t), n(t)\}$
		\item Si $c$ est de vitesse 1 alors $c$ biréguliére $\Leftrightarrow$$||\ddot c (t)||\neq 0,\ \forall t$ car dans ce cas $\expval{\dot c(t),\ddot c(t)}=0$ donc $||\dot c()\wedge \ddot c (t)||=||\dot c(t)||\cdot ||\ddot c(t)||\neq 0$ de plus $κ(t)=||\ddot c(t)||$ (car $κ(t)=\expval{\dot ν,n(t)}=\expval{\dot c(t),\frac{\ddot c(t)}{||\ddot c(t)||}}=||\ddot c(t)||$).
		\item En particulier pour une courbe dans l'espace $κ(t)\geq0\ \forall$
		\item Si $c(I)=imc\subset plan\subset\R^3$ la courbure de $c$ n'est pas même que la courbure definie pour la xstihon $\hat c$ au plan on a $κ=|\hat κ|$.
		\item Ce plan osculateur est indipendant de la parametrisation. $\check c=c\circ φ; \dot\check c=\dot c\circ φ \cdotφ; \ddot\check c=\ddot c\circ φ \dot φ^2+\dot c°φ \ddot φ$.
		($\vect\{\dot\check c(t),\ddot \check c(t)\}=vect\{\dot c(\phi(t)),\ddot c(\phi(t))$).
	\end{enumerate}
\end{remark}

\begin{proposition}
	Equations de Frenet pour une courbe biréguliére.
	\begin{align*}
		\dot ν (t) &= &\frac 1{||\dot c(t)||}κ(t)n(t)\\
		\dot n(t) &= &\frac 1{||\dot c(t)||}(-κ(t)ν(t)+τ(t)b(t))\\
		\dot b(t) &= -&\frac 1{||\dot c(t)||}τ(t)n(t)
	\end{align*}
	Memo $\begin{array}{c}ν\\n\\b\end{array}=\begin{array}{ccc}0&0&0\\0&0&0\\0&0&0 \end{array}\begin{array}{c}\end{array}$.
\end{proposition}
\begin{proof}
	$κ=\frac 1{||\dot c||}\expval{\dot ν,n}=0\ (1)$\\
	$\expval{\dot ν,b}=0$ car $\dot ν\in\vect\{\dot c, \ddot c\}$. $\expval{ν,b}=0$ $\Rightarrow$$\expval{\dot ν,b}+\expval{ν,\dot b}=0$ donc $\dot b\perp ν$.
	$τ=\frac 1{||\dot c||}\expval{\dot n, b}$ $\expval{n,b}=0$ $\expval{\dot n,b}+\expval{n,\dot b}=0$ $\Rightarrow$(3). (2) découle donc de $\expval{\dot n, ν}=-\expval{n,\dot ν}$ car $\expval{ν,n}=0$ $\expval{\dot n, b}$ definition de $\tau$.
\end{proof}

\begin{theorem}
	[foundammentale de la théorie de Frenet]
	Soit $I$ un intervalle et $κ,τ\in C^∞(I,\R)$, $κ(t)\geq 0$. Alors il existe une courbe paramétrie de vitesse 1 $c\in C^∞(I;\R^3)$ tq. sa courbure et sa torsion sont $τ$ et $κ$. Toute autre courbe qui ales mémes propriétés est de la forme: $\hat c=F\circ c$ avec $F(x)=Ax+b$ avec $A\in SO(3)$.
\end{theorem}
\begin{proof}
	Ce systéme d'équations differentielles:
	\begin{align*}
		\dot ν &= κn\\
		\dot n &= -κ\nu + τb\\
		\dot b &= -τn
	\end{align*}
	est lineaire et d'ordre 1. Pour tout systeme orthonue diuct et $\forall t_0\in I:\ \{e_1, e_2, e_3\}$ il existe une solution t.q.
	\begin{align*}
		ν(t_0)&=e_1\\
		n(t_0)&=e_2\\
		b(t_0)&=e_3
	\end{align*}
	on define $c(t_0)+∫_{t_0}^tν$ pour un $c(t_0)\in\R^3$
\end{proof}

\begin{examplebox}[Pour courbure et $\bar c$osion]
	$κ=\frac 1{||\dot c||}\expval{\dot ν, n};\ τ=\frac 1{||\dot c||}\expval{\dot n, b}$.
	Soit
	\begin{align*}
		c(t) &:=(\cos t, \sin t, t),\ t\in\R\\
		\dot c(t)&=(-\sin t, \cos t, 1);\ ||\dot c(t)||^2=2\\
		\ddot c(t)&=(-\cos t, -\sin t, 0)\\
		ν(t)&=\frac 1{\sqrt{2}} (-\sin t, \cos t, q)\\
		b(t)&=\frac{\dot c \wedge  \ddot c}{||\dot c \wedge  \ddot c||}(t)=\frac{(\sin t,-\cos t, 1)}{\sqrt{2}}\\
		n(t)&=-(\cos t, \sin t, 0)\\
		\dot ν(t)&=\frac 1{\sqrt{2}}(-\cos t, -\sin t, 0)\\
		& \expval{\dot ν, n}=\frac 1{\sqrt{2}} \Rightarrow κ=1\\
		\dot n(t)&=-(-\sin t,\cos t, 0)\\
		& \expval{\dot n, b}=\frac1{\sqrt2} \Rightarrow τ=1
	\end{align*}
	Image
\end{examplebox}

\begin{remark}[Theoreme foundamentalle dans le plan]
	Soit $κ\in C^∞(I;\R)$ pour un intervalle $I$. Alors il existe une courbe paramétrie par lagueur d'arc $c$ t.q. sa courbure est $κ$. Toute autre courbe set un $\hat c$ avec les mêmes proprietes est de forme:
	$$\hat c(t)=F\circ c(t+t_0),$$
	pour $t_0\in \R$ et F une isometrie directe $\Leftrightarrow$deplacement.
\end{remark}

Deux résultats sur la géométrie globale des courbes dans l'espace.

\begin{definition}[courbure totale]
	Soit $c\in C^∞(\R;\R^3)$ une courbe paramétrie par longueur d'arc et périodique de période L, $κ\in C^∞(I;\R)$ est sc courbure. Alors $κ(c):=∫_0^Lκ(t)\dd{t}$ est appelé \textsc{courbure totale} de $c$. 	
\end{definition}

\begin{remark}
	Dans le p'au on sait (Hopf) que $κ(c)=±1$ si $c$ est simple.
\end{remark}

On peut dénoutrer
\begin{theorem}[Fenchel]
	Soit $c\in C^∞(\R;\R^3)$ une courbe fermée simple. Alors pour sa courbure totale:
	$$κ(c)\geq2π.$$
	De plus on a $κ(c)=2π$ $\Leftrightarrow$$c$ est un courbe plane et convexe.
\end{theorem}
\begin{proof}
	Sans.
\end{proof}
On peut dénouter
\begin{theorem}[Fary-Tlilnor]
	Soit $c\in C^∞(\R;\R^3)$ une courbe fermée simple. Si $c$ admet un \texttt{noeud} alors pour la courbure totale on a
	$$κ(c)≥4π.$$
\end{theorem}
\begin{remark}
	Si $c$ admet un \texttt{noeud}, c'est à dire on ne peut définir $c$ d'une manière continue en une courbe plane fermée simple.
\end{remark}

\begin{definition}
	Une \textsc{Isotopie} de $\R^3$ est une application.
	$$φ\in C^0([0,1]\times\R^3;\R^3)$$
	t.q. $\forall t\in[0,1]\ φ(t,•)$ est un \texttt{homeomorphism}.
\end{definition}
\begin{definition}
	Deux courbes fermeies simples $c_1, c_2$ sontnt appélé \textsc{Isotope}. S'il existe une isotopie $φ$ t.q.
	$$φ(0,X)=X\ \forall x\in\R^3;\ φ(1, \img(c_0))= \img(c_1).$$
\end{definition}
\begin{definition}\leavevmode
	\begin{itemize}
		\item Un noeud est une class l'equivalence d'une isotopie.
		\item Une courbe fermé simple est \textsc{Sans Noeud}, si elle est isotope à une courbe plane fermée simple.
	\end{itemize}
\end{definition}

\ifcomment
$\Leftrightarrow$ \Leftrightarrow 
\pdv  \pdv
\pd_{x_2} \pd_{x_2}
\mapsto  \mapsto
\rightarrow \rightarrow
\fi

\section{surfaces}

\begin{definition}[Surface régulière]
	Soit $S\subset \R^3$. $S$ est appelé \textsc{Surface Régulière}. Si pour chaque $p\in S$ il existe un ouvert $V\subset \R^3$ t.q. $p\in V$ et s'il existe un ouvert $U\subset\R^2$ et un $F:\underbrace{U}_{\subset \R^2}\rightarrow\R^3$ $C^∞$ t.q. 
	\begin{enumerate}
		\item $F(U)=S\cap V$ et $F:U\rightarrow S\cap V$ est un homéomorphisme (c.a.d. $F|_U$ continue et son inverse $F\dmo|_U$ est continue)
		\item Le Jacobien $Du F$ a $\rank 2$ $\forall u\in U$
	\end{enumerate}
\end{definition}
\begin{remark}
	La matrice jacobienne dans U repère standard:
	$$F(X_1, X_2)=(F_1(X_1, X_2),F_2(X_1, X_2),F_3(X_1, X_2))$$
	$$Du J = 
	\mqty(
	\pd_{x_1} F_1 & \pd_{x_2} F_1\\
	\pd_{x_1} F_2 & \pd_{x_2} F_2\\
	\pd_{x_1} F_3 & \pd_{x_2} F_3
	)$$
	
	$U=(x_1, X_2)$
	$\pd_{x_j} F = \mqty(
	\pd_{x_j} F_1\\
	\pd_{x_j} F_2\\
	\pd_{x_j} F_3
	)$
	
	donc rang $Du F=2$ $\Leftrightarrow$  $\pd_{x_1} F, \pd_{x_2} F$ sont indépendants $\dim \vect\{\pd_{x_1} F, \pd_{x_2} F\}=2$ $\Leftrightarrow$ deux vecteurs tangents à $S$ au point $F(u)$ qui sont indépendant c'est a dire : on peut définir l'espace tangent $\Leftrightarrow$ $||\pd_{x_1} F\wedge\pd_{x_2} F||\neq 0$.
	
	$u_1=(x_1,x_2)$ la ligne $x_2=\ct{const}$ qui passe par $U$. $\R\ni t\mapsto  (x_1,x_2+t)=:c(t)$, $c(0)=u$. $t\mapsto  F(c(t))$ est la courbe correspondante sur $S$.
	
	$\pdv  F(c(t))|_{t=0}=\pdv  F(x_1,x_2+t)|_t=0 = \pd_{x_2} F(x_1, x_2)$
\end{remark}

\begin{definition}
	Pour une surface régulière l'application $F:U\rightarrow S\cap V$ (on encore $(U,F,V)$) \textsc{Paramétrisation Locale} de $S $au point $p$.
	$S\cap V$ est appelé un \textsc{Voisinage de Coordonnées} et les composantes $(u_1,u_2)$ de $u$ t.q. $F(u)=p$ les \textsc{Coordonnées de $p$ par Rapport} à $F$.
\end{definition}
\begin{example}
	Pour $p\in \R3$ et $X_1$, $X_2\in \R^3$ le plan affine $S:=\{X, X=p+u_1X_1+u_2X_2 \}$ est une surface régulière. Car: On peut prendre (pour tout $p\in S$) $V:=\R^3; U:=\R^2$
	$F(u_1, u_2)=p+u_1X_1+u_2X_2$
	
	F es une fonction affine donc F est différentiable. (en tout que fonction de $\R^2\rightarrow\R^3$)
	$F(U)=S=S\cap\R^3 F:U\rightarrow S$ est un homéomorphisme.
\end{example}
\begin{example}[graphe d'une fonction]
	(Une seule paramétrisation!) Soit $U\subset \R^2$ ouvert $f:U\rightarrow\R$ différentiable. $S=\{x=(x_1,x_2,x_3): (x_1,x_2)\in U, x_3=f(x_1, x_2)\}$
	
	On peut prendre de nouveau $V=\R^3$ $U$ (est $U$)
	$F(u_1,u_2):=(u_1, u_2, f(u_1, u_2))$ $F:U\rightarrow\R^3$ est différentiable. $F:U\rightarrow F(U)=S$ est continue $F|_n\dmo$ est la projection orthogonale donc continue. La surface est régulière car 
	$\pd_{u_1}F=(1,0,\pd_{u_1}f(u_1,u_2))$
	$\pd_{u_2}F=(0,1,\pd_{u_2}f(u_1,u_2))$
	$\pd_{u_1}\wedge\pd_{u_1}= (.,.,1)\neq 0$
	
	Addendum: le plan affine est régulier
	$X=p+u_1X_1+u_2X_2$
	$\pd_{u_1}F=X_1$, $\pd_{u_2}F=X_2$
	$\pd_{u_1}F\wedge\pd_{u_2}F=X_1\wedge X_2\neq Si X_1,X_2$ sont indépendantes $\Leftrightarrow$ $\dim \vect\{X_1, X_2\}=2$.
\end{example}

\begin{example}
	$S(=S^2)=\{(x,y,z)\in \R^3; x^2+y^2+z^2=1\}$
	$S$ est une surface régulière?
	Soit $p=(p_1,p_2,p_3)\in S$ t.q. $p_3>0$
	$F(X,Y)=(X,Y, \sqrt{1-x^2-y^2})$ ($x^2+y^2<1$)
	$U:=\{(X,Y);\ x^2+y^2<1\}; V:= \{(x,y,z); z>0\}$
	
	$S\cup V_3$ est le graphe
	de $(X,Y)\mapsto \sqrt{1-x^1-y^2}$ qui est $C^∞$ par l'exemple du graphe on a que $F$ est une paramétrisation en $p $pour chaque $p\in S\cap V_+$
	
	Soit $p\in S$; $p_3<0$
	on choisi $U:=\{(x,y); x^2+y^2<1\}$ $V_-=\{(x,y,z); z<0\}$
	$F_-(x,y):=(x,y,-\sqrt{1-x^2-y^2})$ $(x,y)\in U$
	$V_-=\{(x,y,z); z<0\}$
	parce que $S\cap V_-$ est le graphe de $U\ni (x,y) \mapsto  -\sqrt{1-x^2-y^2}$ qui est différentiel. Par le précédent $(U,F_-, V_-)$ est un voisinage de coordonnées pour chaque point $p\in S$ t.q. $p_3<0$.
	
	$\{p\in S\text{ t.q. } p_2>0 \}$ est le graphe $U\in (x,y)\mapsto \sqrt{1-y^2-z^2}$ donc par le précédent $(U,F_{1_±}, V_{1_±})$ avec $V_{1_±} ={(x,y,z), x>_<0}$ et $F_{1_±}=(y,z,±\sqrt{1-y^2-x^2})$
	De même: $(U, F_{2_±}, V_{2_±})$ avec $V_{2_±}=\{x,y,z x>0\ y<0\}$ $F_{2_±}(X,z)=(x,z,±\sqrt{1-x^2-z^2})$ est un voisinage de coordonnées pour $\{p\in S; p_2>_<0\}$
	
	En résumé: $S^2$ est une surface régulière.
\end{example}

\begin{remark}
	Il nous a falloir 6 paramétrisations pour monter que $S$ est la une surface régulière. On peut faire avec 2 paramétrisations mais pas avec 1.
\end{remark}	
	
\begin{proposition}
	Soit $V_0\subset\R^3$ ouvert $f\in C^∞(V_0; \R)$
	$S:=\{ (x,y,z)\in V_0; f(x,y,z)=0\}$
	Si $\grad f(p)\neq 0 \forall p\in S$ alors $S$ est une surface régulière.
\end{proposition}
\begin{remark}
	\begin{itemize}
		\item $S^2=f\dmo (0)$ pour $f(x,y,z)=x^2+y^2+z^2 -1$
		\item $S$ --- le plan affine $=f\dmo (0)$ de $f(X)=\expval{X-P,n}$ pour un $p\in S$ et $n$ un vecteur normale à $S$.
	\end{itemize}
\end{remark}
\begin{proof}
	Soit $p=(X_0,Y_0,Z_0)$
	$grad f(p)=(\pd_x f(p),\pd_y f(p),\pd_z f(p))\neq (0,0,0)$
	
	Supposons que $\pd_z f(p) \neq 0$. Par le théorème des fonctions implicites il existe un voisinage $V\subset V_b$ de $p$ un voisinage $U\subset \R^2$ de $(X_0,Y_0)$ et une fonction $g\in C^∞ (U,\R)$ t.q. $S\cap V=\{(x,y,g(x,y)); x,y\in U\}$ donc on conclure en utilisant l'exemple du graphe d'une fonction (cad $f(x,y,g(x,y))=0$).
\end{proof}

Attention: la condition $\grad f(p)\neq 0 (p\in S)$ est suffisante mais pas nécessaire. Par exemple $S^2=\tilde f\dmo (0)$
pour $\tilde f(x,y,z)=(x^2+y^2+z^2-1)^2$ $\grad \tilde f(x,y,z)=2(x^2+y^2+z^2-1)2(x,y,z) =0$ si $x^2+y^2+z^2=1$
\begin{example}
	$f(x,y,z)=x^2+y^2-z^2 (x,y,z)\in\R^3$
	$S=f\dmo(0)$
	$\grad f(x,y,z)=2(x,y,-z)=0$ $\Leftrightarrow$ $(x,y,z)=(0,0,0) (0,0,0)\in S$
	
	In faut donc examine $S$ autour (=dans un voisinage) de $(0,0,0)$
	$S=\{ (x,y,z); |z|=\sqrt{x^2+y^2}\}$
	
	$S$ est un double-cône 
	\begin{remark}
		rotation de la courbe $X\mapsto  (X,Z)$ avec $|x|=|y|$ autour de l'axe des $z$
	\end{remark}
	Il ne eut exister de voisinage $V\subset\R^3$ de $(0,0,0)$ et $U\subset \R$ ouvert t.q. $F|_U:U\rightarrow S\cap V$ soit homeomorphe avec $F:U\rightarrow\R^2$ t.q. $Du F$ est de $rang 2$ 
	car pour $p\in S\cup V$ avec $p_3>0$ et $q\in S\cap V$ avec $q_3< 0$ et toute courbe $c:[0,1]\rightarrow S\cap V $avec $c(0)=p$, $c(1)=q$. $\exists t_0$ t.q. $c(t_0)=(0,0,0) $
	or dans $U$ il existent des courbes qui évitent l'origine. C'est à dire $γ\in C^0([0,1], U)$ $γ(0)=F\dmo(q)$ $γ(1)=F\dmo(p)$ $γ(t)\neq F\dmo(0) \forall t\in[0,1]$.
\end{example}


\begin{proposition}
	$S\subset \R^2$ surface régulière et $(U,F,V)$ une paramétrisation Loz- a $U$. Soit $W\subset\R^n$ ouvert et $\phi:W\rightarrow \R^3$ t.q. $φ(W)\subset S\cap V$ alors $φ\in C^∞(W;\R^3)$ $\Leftrightarrow$  $F\dmoºφ\in C^∞(W,U)$.
\end{proposition}
\begin{proof}\ \\
	($\Leftarrow$ ) $φ=\underbrace{F}_{\C^∞} º\underbrace{(F\dmoºφ)}_{\C^∞}$
	($\Rightarrow$ ) Soit $φ$ différentielle. On sait que rang $D_u f=2$
	
	$$D_u f \cong  \mqty(\pd u_1 F_1 & \pd u_2 F_1 \\ \pd u_1 F_2 & \pd u_2 F_2 \\\pd u_1 F_3 & \pd u_2 F_3)$$
	
	Supposons qui  $\det\mqty(\pd u_1 F_1 \pd u_2 F_1\\ \pd u_1 F_2 \pd u_2 F_2)≠0$.
	
	Soit $G:U\times \R\rightarrow \R^3$.
	$G(u_1,u_2, T):= F(u_1,u_w)+(0,0,t)=(F_1(u_1,u_2,F(u_1,u_2), F(u_1,u_2)+t))$ alors $G$ est différentiable, 
	$D_{(u_1u_2t)}G\cong \mqty(\pd u_1 F_1&\pd u_2 F_1&0\\\pd u_1 F_1&\pd u_1 F_1&0\\\pd u_1 F_1&\pd u_1 F_1&1)$
	donc $\det D_{(u_1u_2t)}G ≠ 0$. Donc par le théorie de la fonction inverse il existe $U_1\subset U\times \R^2 et V_1\subset V$ t.q. $G|_{U_1}:U_1\rightarrow V_1$ est un difféomorphisme. Soit $W_1=φ\dmo(V_1)$ pour $\hat p\in W_1$ on $G\dmoºφ(\hat p)=(F\dmoºφ(\hat p), 0)$ car $F(u_1,u_2)=G(u_1,u_2, 0)$ $G\dmoºφ$ est $C^∞$ sur $W_1$.
\end{proof}

\begin{corollary}
	Soit $S$ une surface régulière et $(U_1,V_1,F_1)$ et $(U_2,V_2,F_2)$ deux paramétrisations locales. Alors $F_2ºF_1\dmo:F_1\dmo(V\cap V_2)\rightarrow F_2\dmo(V_1\cap V_2)$ est $\C^∞$.
\end{corollary}
\begin{proof}
	On applique la proposition précédente à $W=F_1\dmo(V_1\cap V_2)$ et $φ=F_1$ et $$(U,V,F):= (U_2,V_2, F_2)$$ 
\end{proof}

\begin{proposition}
	Soit $S\subset\R^3$ une surface régulière $F:S\rightarrow \R^n$ continue. Soit $p\in S$, alors sont équivalents:
	\begin{enumerate}
		\item $\exists V\subset\R^3$ voisinage de $p$ et une extension $\hat f$ de $f|_{S\cap V}$ à $V$
		\item $\exists$ une paramétrisation locale $(U,F,V)$ avec $p\in V$ t.q. $fºF:U\rightarrow \R^n$ est $C^∞$
		\item $\forall$ paramétrisation locale $(U,F,V)$ avec $p \in V$ $fºF: U\rightarrow  \R^n$ est $C^∞$
	\end{enumerate}
\end{proposition}
\begin{proof}
	\begin{enumerate}
		\item (1) $\Rightarrow$  (2)
		Car $F$ est $C^∞$ $fºF=\hat fºF$
		\item (3) $\Rightarrow$  (2) Ok
		\item (2) $\Rightarrow$  (1) 
		On considère (de nouveum) $(U_1, U_2, t):= F(u_1,u_2)+(0,0,t)$ Soit $g(u_1,u_2):=fºF(u_1,u_2)=fºG(u_1,u_2,0$)
		donc $g$ est $C^∞$ en $(F\dmo(p),0)$ et $\hat f:=gºG\dmo$.
	\end{enumerate}
\end{proof}
\begin{definition}
	Soit $S\subset \R^3$ une surface régulière et $f:S\rightarrow \R^n$ continue. On dit que $f$ est $C^∞$ \textsc{en} $p$ si une des trois assertions équivalentes du précédent est vraie.
\end{definition}
\begin{definition}
	Soit $S_1,S_2\subset\R^3$ deux surfaces régulières. Soit $f:S_1\rightarrow S_2$ continue On dit que $f$ est $C^∞$ en $p\in S_1$. Si'l existe une paramétrisation locale $(U_1,V_1, F)$ de $S_1$ en $p$ et une paramétrisation locale $(U_2, V_2, F_2)$ de $S_1$ en $f(p)$ t.q. $F_2\dmoºfºF_1:F\dmo_1(f\dmo(V_2)\cap V_1)\rightarrow  U_2$ est $C^∞$ en $p$.
\end{definition}
\begin{remark}
	Si $F_2\dmoºfºF_1 est C^∞$ pour deux paramétrisations alors $\tilde F_2\dmoºfº\tilde F_1$ est $C^∞$ pour toutes paramétrisations $\tilde F_1, \tilde F_2$. Car $F_1\dmoº\tilde F_1$ et $F_2\dmoº\tilde F_2$ sont $C^∞$ par le précédent.
\end{remark}
\begin{corollary}
	Soient $S_1, S_2$ deux surfaces régulières et $V\subset\R^3$ ouvert t.q. $S_1\subset V$ . Soit $f:V\rightarrow \R^3$ t.q. $f(S_1)\subset S_2$ alors $f|_{S_1}:S_1\rightarrow S_2$ est différentiable.
\end{corollary}
\begin{definition}
	Soit $S_1,S_2$ deux surfaces régulières. $f:S_1\rightarrow S_2$ est appelé \textsc{Difféomorphisme} si $f$ est bijection et si $f$ et $f\dmo$ sont différentiables. Dans ce cas on dit que $S_1$ est \textsc{Difféomorphe} à $S_2$.
\end{definition}
\begin{examplebox}
	Soit $S_2 = S^2$ (la sphère) et $S_1\rightarrow S_2=\{(x,y,z)\in \R^3; \frac{x^2}{a^2}+\frac{y^2}{b^2}+\frac{z^2}{c^2}=1\}$ pour $a,b,c>0$ e'ellipsoïde $S_1$ est une surface régulière car $S_1=g\dmo(0)$ pour $g(x,y,z)=\frac{x^2}{a^2}+\frac{y^2}{b^2}+\frac{z^2}{c^2}-1$  $\grad g(x,y,z)≠(0,0,0)$ pour $(x,y,z)\in S_1$. Soit $f:S_1\rightarrow S_2$ $f(x,y,z):=(\frac xa,  \frac yb, \frac zc)=(f_1,f_2,f_3)$ est bien définie car $(f_1^2+f_2^2+f_3^2)(x,y,z)=1$ pour $(x,y,z)\in S_1 f\dmo(x,y,z)=(ax,by,cz)$ $f$ et $f\dmo$ sont continue et $C^∞$. en tant que fonctions de $\R^3\rightarrow \R^3$ donc $f$ est un difféomorphisme et l'ellipsoïde et la sphère sont difféomorphes.
\end{examplebox}
\begin{example}
	$φ:U\subset\R^2\rightarrow \R\ φ\in C^∞$ alors le graphe de $φ$ est difféomorphe à $U\times \{0\}$ car $(x,y,0)\mapsto (x,y,φ(x,y))=f(x,y)$ est un difféomorphisme. $f\dmo(p_1,p_2,p_3)=(p_1,p_2)$.
\end{example}
\begin{definition}
	Soit $S\subset \R^3$ une surface régulière. L'espace tangent à $S$ au point $p\in S$ est 
	$$T_pS:=\{X\in\R^3; \text{il existe une courbe $c\in C^∞((-ε,ε));S)$ t.q. $c(0)=p$ t.q. $\dot c(0)=X$} \}$$
\end{definition}
\begin{proposition}
	Soit $S\subset \R^3$ une surface régulière, $p\in S $ et $(U,F,V)$ une paramétrisation en $p$. Soit $u_0=F\dmo(p)$ alors $T_pS=
	\text{ « image de $(D_{u_0} F)$ » }= \vect\{\pd_1 F(u_0), \pd_2 F(u_0)\}$.
\end{proposition}
\begin{proof}
  	($\supset$) Soit $X \in \text{ « image $D_{u_0}F$ » }$ et $Y\in \R^2$ t.q. $D_{u_0}F(Y)=\pd_yF(u_0)=\pdv{t} F(u_0+ty)|_{t=0}=X$
	Soit $c(t):=F(u_0+ty)\ t\in(-ε,ε)$ pour $ε$ suffisamment petit pour $(-ε,ε)\ni t\mapsto u_0+ty'\in U$.
	Donc $\pdv{t} c(t)|_{t=0}=D_{u_0}F(y)$
	($\subset$) Soit $X\in T_pS$ et $c$ t.q. $\dot c(0)=X$. Soit $u(t):=F\dmo º c(t) u\in C^∞$ en $u_0$ (parce $c $est $C^∞$). Soit $y:=\dot u(0)$. Alors $D_{u_0}F(y):=\pdv{t} Fºu(t)|_{t=0}=\pdv{t} c(0)$.
\end{proof}
\begin{corollary}
	$\dim T_p S=2$
\end{corollary}
\begin{proof}
	$T_pS=\image D_{u_0} F,\ \rank D_{u_0}F=2\ \forall u_0$
\end{proof}
\begin{proposition}
	Soit $V\subset \R^3$ est ouvert $f:V\rightarrow \R^3$ une fonction $C^∞$ $S:=f\dmo(0)$ $\grad f(X)≠0$ si $f(x)=0$
	
	Alors pour $p\in S$
	$T_pS=[\grad f (p)]^\perp=\{x\in\R^3; \expval{X-p,\grad f(p)}=0$.
\end{proposition}
\begin{proof}
	$t\mapsto c(t)\in S$; $X=\dot c (0)$ $\Rightarrow$  $fºc(t)=0$;
	$\pdv{t} fºc(t)|_0=D_{c(0)}f(\dot c(0))=\expval{\grad f(c(0)), \dot c (0)}=\expval{\grad f(c(0)),X}=0$.
\end{proof}
\begin{examplebox}
	$p\in S^2=\{x\in\R^3;\norm{x}=1\}$
	$T_pS^2=p^\perp \grad (X\mapsto \norm{X}^2-1)=2X$
\end{examplebox}
\begin{definition}
	Soient $S_1, S_2$ deux surfaces régulières $f\in C^∞(S_1,S_2)$. Alors $p\in S_1$ la dérivée de $f$ est définie par $d_pf:T_pS_1\rightarrow T_{f(p)}S_2$. définie telle que pour $X\in T_pS_1 X=\dot c(0)$ pour $c\in C^∞((-ε,ε);S^1)$; $p=c(0)$ car $d_pf(X):= \pdv{t} fºc(0) \in T_{f(p)}S_2$.
\end{definition}

\begin{proposition}
	*La définition de $f(X)$ ne dépend pas de la courbe c qu'on utilise pour définir $d_p f(X)$\\
	* $d_p f:t_pS_1\rightarrow T_f(p)S_2$ est linéaire.
\end{proposition}
\begin{proof}
	$\hat f=F_2\dmoºfºF_1$. Soit $u_0=F_1\dmo(p)$. Soit $a(t):=F_1\dmo(x(t))$ donc $D_{u_0}F_1(\dot u(0))= \dv{t} c(t)|_{t=0}$
	$d_p f(x)=\dv{t} fºc(t)|_{t=0}=\dv{t} F_2–f_2\dmoºfºF_1(u(t))|_{t=0}=D_{u_0}(F_2º\hat f)(D_{u_0}F_1)\dmo(X)$.
	
	Donc $d_pf(x)=Du0 (F_2º\hat f)((Du0 F_1)\dmo(X)) $la membre de droite est indépendant de $c$ et linéaire.
	
\end{proof}
\begin{remark}
	$d_pf$ est donc essentiellement déterminé par le Jacobien de $\hat f$.
	
	cad \emph{diagramme} est un diagramme commutatif.
\end{remark}

La première forme fondamentale d'une surface régulière $S$ au point $p$ la restriction du produit scalaire euclidien de $\R^3$ sur $T_pS$
\begin{definition}
	Soit $S$ une surface régulière et $p\in S$.
	$g_p:T_pS\times T_pS\rightarrow \R$
	$g_p(X_p,Y_p):=\expval{X_p,Y_p}$
\end{definition}
\begin{remark}
	Soit $(U,F,V)$ une paramétrisation en p alors on peut exprimer la forme bilinéaire $g_p$ par une matrice. $((g_{ik}(p)))$
	$T_pS=\vect\{\pd_1F(p),\pd_2F(p)\}$
	$((g_{ik}(p)))=\mqty(g_p(\pd_1F(p),\pd_1F(p))& g_p(\pd_1F(p),\pd_2F(p))\\g_p(\pd_2F(p),\pd_1F(p))&g_p(\pd_2F(p),\pd_2F(p)))$
	donc pour
	$X=X^1\pd_1+X^p\pd_F$ avec $X^1, X^2 \in\R$
	$Y=∑_{j=1}^2 Y^j\pd_jF$ alors $g_p(X_p,Y_p)=∑_{j,k}^2g_{ik}(p)X^j(p)Y^k(p)$
\end{remark}
\begin{remark}
	$S\ni p\mapsto  g_p$ est une fonction à valeurs dans les formes bilinéaires tenseur covariance de degré $2$.
\end{remark}
\begin{example}
	$X_1,X_2\in \R^2$ indépendantes \\
	A) $S=\vect\{X_1,X_2\} X=∑_jX^jX_j$; $Y=∑_kY^kX_k$
	$\expval{X,Y}=∑X^jY^k\expval{X_i,X_k}$\\
	Ex: $X_1 = \vc(1\\1\\0), X_2=\vc(0\\3\\0)$ alors $g_{jk}=\mqty(2&3\\3&9)$
	B) $F:(0,∞)\times (0,2π)\rightarrow \R^3 F(r,φ)=(r\cos φ,r\sin φ,0)$
	$\pd_1 F(r, φ)=\mqty(\cos φ&\sin φ& 0)$
	$\pd_2 F(r, φ)=\mqty(-r\sin φ&r\cos φ& 0)$
	$((f_{ik}))=\mqty(1 &0 \\0& r^2)$\\
	C) Ex 1(ii)
	$F(x,y)-(\cos x\cos y,\cos y\sin y,\sin x) x\in(-π/2,π/2)\times \R$
	$F$ est une paramétrisation de 
	$S^2\diagdown \{N,S\} N=north, S=south$
	$((g_{ik}(x,y)))=\mqty(1&0\\0&\cos^2x)$
\end{example}

Champ normal, application de Gauss, Orientabilité

\begin{rappel}
	Définition de la courbure pour une courbe plane (de vitesse 1)
	$κ=\expval{\dot v, n}=-\expval{\dot n, v}$
	$ν$-tangente $n$-normale
	La courbure est donc la variation de la normal dans la direction de la tangente.
\end{rappel}
\begin{definition}
	Soit $S$ une surface régulière un champs normal sur $$S$$ est une application $$N:S\rightarrow \R^3$$ t.q. $$N(p)\perp T_pS \forall p\in S$$.
\end{definition}
\begin{example}
	A)
	$$S=\R^3\times\{0\}$$ alors $$S\ni p\mapsto  N(p)=(0,0,1)$$ est un champs normal unitaire.
	B) $$S^2=\{p\in\R^3:\ \norm{p}=1\}$$
	$$N(p)=p$$ est un champs normal unitaire $$\norm{n(p)}(f) fgd= gdf1$$
	$$N(p)= 2p$$ est un champs normal $$N(p)=g(p)p$$ est un champs normal pour $$g(p):S^2\rightarrow \R$$.
	C) le ruban de Môbius n'admet pas de champs normale unitaire continue.
\end{example}
\begin{definition}
	Une surface régulière appelé orientable s'il existe un champs normale unitaire différentiable.
\end{definition}
\begin{remark}
	Chaque surface est localement orientable c'est à dire: Soit $$(U,F,V)$$ une paramétrisation. $$\pd_1F\wedge\pd_2F≠0$$
	alors
	$$u\ni p\mapsto \frac{\pd_1F(p)\wedge\pd_2F(p)}{\norm{\pd_1F(p)\wedge\pd_2F(p)}}\in\R^3 $$
	est bien définie (régularité) et différentiable.
	Soit $$(U_2,F_2,V_2)$$ une deuxième paramétrisation.
	$$N_2(p)=\frac{\pd_1F(p)\wedge\pd_2F(p)}{\norm{\pd_1F(p)\wedge\pd_2F(p)}}|_{V=F\dmo_2(p)}$$
	
	$$N_1(p)=±N_2(p) =\det F_2\dmoºF_1(u)n_2(p)^r$$
	$$Ax\wedge Ay=\det A A (x\wedge y)$$
	$$\pd_jF_DF(e_j)$$
	$$DF_2ºF_1=DF_2ºF_1DF$$
\end{remark}
\begin{theorem}
	Une surface régulière est orientable si et seulement si; it existe un recouvrement par des paramétrisation $(U_j, F_j, V_j)$ telle que $\det D(F_j\dmoºF_k)>0$ pour tout $j,k$.
\end{theorem}

La deuxième forme fondamentale

\begin{remark}
	Une application $A:\R^2\rightarrow \R^2, A=A^T$ \\
	* peut être caractérise complètement par la forme bilinéaire $(X,Y)\mapsto \expval{X,AY}$
	* peut être degonalisé ($\Leftrightarrow$  valeurs propres vecteur propres)\\
	* on va considère $d_pN$
\end{remark}
\begin{definition}
	Soit $S$ une surface régulière orientable et $N:S\rightarrow S^2\subset\R^3$ un champs normale unitaire différentiable.
	$N$ est appelée une application de Gauss.
	L'endomorphisme $W_p:T_pS\rightarrow T_pS X\mapsto W_p(X):=-d_pN(X)=-dp X N(p)$ --- drive directionnelle. est appelé l'application de forme (shape orientator) ou application de Weingarten.
\end{definition}
\begin{remark}
	$d_pN(X)\in T_pS$ car $N(p)\perp T_pS \norm{N(p)}=1$ donc $\expval{d_pN(X),N(p)}=0$
\end{remark}
\begin{example}
	A) $S=plan N(p)=(0,0,1)$ $\Rightarrow$ $p_pN=0$
	B) $S^2$ la sphère $N(p)=p d_pN=1$
\end{example}
\begin{proposition}
	Soit $S$ surface régulière orientable.
	L'application $W_p$ est une application symétrique par rapport à la première forme fondamentale, c'est a dire $g(X,WY)=g(WY,X)$ c'est à dire, $\forall p\in S; X_p, Y_p\in T_pS$. $\expval{X_p,W_pY_p}=\expval{W_pX_p,Y_p}$
\end{proposition}
\begin{proposition}
	La forme bilinéaire $h_p:TpS\times T_pS\rightarrow \R$
	i) $h_p(X_p,y_p):=g_(X_p,W_pY_p)=\expval{X_p,W_p(Y_p)}=-\expval{X_p,d_pN(Y_p)}$	est appelée la deuxième forme fondamentale de $S$ en $p$.\\
	ii) $W_p$ est appelée diagonalisable. les vecteurs propres de $W_p$ sont appelés directions principales les valeurs propres sont appelées les courbures principales.\\
	iii) $p\mapsto \det W_p$ est appelée la courbure de Gauss.\\
	iv) $p\mapsto 1/2 trace W_p$ est appelée la courbure moyenne.
\end{proposition}
\begin{proposition}
	$(U,F,V)$ une paramétrisation dans la base $\{\pd_1F(p),\pd_2F(p)\}$ de $T_pS$
	$g_{jk}(p):=\expval{\pd_jF,\pd_kF}$ sont les coefficients de la première forme fondamentale.$h_{jk}(p)=\expval{\pd_{jk}F,N}(p)$ sont les coevidientd de deuxième forme fondamentale $((W))$ le matrice de $W_p$ est donné par $((W))=((g))\dmo((h))$
	
	$w_{jk}=∑_eg\dmo_{je} h_{ek}$\\
	*$k(p)=\det W_p=\frac{\det((W))}{\det((g))}$
	$H(p)=\frac 12 \tr((g))\dmo((h))$
\end{proposition}

En chaque pt de $u$, le vecteur normal à $M$ en $p=F(u)$. $n(u)$, $\norm{n(u)}=1$, $\expval{n(u),e_i(u)}=0$.

$n(u)=±\frac{e_1(u)\wedge e_2(u)}{\norm{e_1(u)\wedge e_2(u)}}$
avec: $\norm{e_1(u)\wedge e_2 (u)}=\norm{\det[\,e_1\ e_2\ \vc{\bar i\\\bar j\\\bar k}]}$.

$M\rightarrow S^2\subset\R^3$. $p=F(u)\mapsto n(u)$.

Application de Gauss de $M$.
$M$ est orientable s'il $\exists$ ensemble de cartes covariant $M$ et une application de Gauss définie globalement sur M que est continue.

Mais le Ruban de Möbius pas orientable.

$\expval{n(u),\pdv{n(u)}{u}}=0$ $\Rightarrow$ $\pdv{n}{u^1},\pdv{n}{u^2}$ sont des vecteurs tangents à $M$ en $p=F(u)$.

$\norm{n(u)}^2=\expval{n(u),n(u)}=1$
$\pdv{u^i}\expval{n(u),n(u)}=0$

$=\expval{\pdv{n(u)}{u^i},n(u)}+\expval{n(u),\pdv{n(u)}{u^i}}
=2\expval{n(u),\pdv{n(u)}{u^i}}=0$

$\pdv{n(u)}{u^i}=W_i^j(u)e_j(u)$

L'application linéaire $W:T_pM\rightarrow T_pM$
$T_pM\ni X=X^ie_i(u)\mapsto W_i^j(u)X^ie_j(u)\in T_pM$ est l'endomorphisme
de Weingarten.

\begin{align*}
	\underbrace{\expval{\pdv{n(u)}{u^i}, e_k(u)}}_{L_{ik}(u)} &= W_i^j(u)\underbrace{\expval{e_j(u),e_k(u)}}_{g_{ik}(u)}\\
	&= W_i^j(u)g_{ik}(u)
\end{align*}

$A=(a_{ij})$, $B=(b_{kn})$

$AB=C=C(c_rs)$ $\Rightarrow$ $C_{rs}=∑a_{rj}b_{js}$

\begin{align*}
	(g^{ij}(u))&=\mqty(g^{11}(u) & g^{12}(u)\\g^{21}(u)&g^{12}(u))\\
	&=(g_{ij}(u))\dmo\\
	&=\mqty(g_{11}(u)&g_{12}(u)\\g_{21}(u)&g_{22}(u))\dmo
\end{align*}

$L=Wg $$\Rightarrow$ $W=Lg\dmo$

$W_i^j(u)=L_{ik}(u)g^{kj}(u)$

$L_{ij}(u)=\expval{\pdv{n(u)}{u^i},e_j(u)}=-\expval{n(u),\pdv{e_j}(u^i)}$

$\expval{V(u),W(u)}=0$

$\pdv{u^i} \expval{\pdv{V}{u^i},W(u)}+\expval{V(u),\pdv{W(u)}{u^i}}=0$

$L(u)=(L_{ij}(u))=\mqty{L_{11}(u)&L_{12}(u)\\L_{21}(u)&L_{22}(u)}
$
2ème forme fondamental de $M$

Si $X,Y\in T_pM, F(u)$,
$X=X^ie_i(u)$, $Y=Y^je_j(u)$.

$L(X,Y)=L_{ij}(u)X^iY^j$ forme bilinéaire sur $T_pM$.

\begin{align*}
	L(X,Y) &= L_{ij}X^iY^j=L_{ji}X^iY^j=L(Y,X)\\
	&=(W_i^kg_{kj})X^iY^j \mbox{ (W est auto-adjoint) }\\
	&=g_{kj}(W_i^kX^i)Y^j\\
	&=\expval{WX,Y}=\expval{XY,X}=\expval{X,WY}
\end{align*}

\begin{definition}
	La courbure moyenne est:
	$\bar κ=\frac 12 \tr(W)=\frac 12 W_i^i=\frac 12 (W_1^1+W_2^2)$ corbure de Gauss.
\end{definition}
$κ=\det(W)$

Les courbures principales $κ_1,κ_2$ valeurs propres de $W$ zéros de $\det(W-ZI)$

$\bar κ=\frac 12 (κ_1+κ_2)$
$Κ=κ_1κ_2$

\section{Courbures normale et geodesique} % (fold)
\label{sec:courbures_normale_et_geodesique}
$M\subset \R^3$ surface régulière $(F,U,V)$ une carte de $M$

$$\pdv{F}{u^i}{u^j}(u)=L_{ij}(u)n(u)+Γ_{ij}^k (u)e_k(u)$$

Formule de Gauss.

Les $Γ_{ij}^k(u)$ sont les symbols de Christoffel de $M$ dans la carte $(F,U,V)$.

$\pdv{F}{u^i}{u^j},\pdv{F}$
% section courbures_normale_et_geodesique (end)




Le paraboloïde hyperbolique:
$F(u,v)=\mqty(u\\v\\-\frac{u^2}{a^2}+\frac{v^2}{b^2})$
$e_u=\pd{F}{u}=\mqty(1\\0\\-\frac{2u}{u^2})$
$e_v=\pd{F}{v}=\mqty(0\\1\\\frac{2v}{b^2})$

$g= \mqty(g_{uu}&g_{uv}\\g_{vu}&g_{vv})$

$Γ_{uuu}=4 \frac u{a^4}$
$Γ_{uuv}=-d\frac v{a^2b^2}$
$Γ_{uvu}=0$
$Γ_{uvv}=0$
$Γ_{vvu}=-4\frac{u}{a^2b^2}$
$Γ_{vvv}=4\frac v{b^4}$

$$g\dmo = \frac{g^T}{1+4(\frac{u^2}{a^4}+\frac{v^2}{b^4}})$$

$Γ_{uu}^u=Γ_{uuu}g^{uu}+Γ_{uuv}g^{vu}=\frac{4u}{Δa^4}$
$Γ_{uu}^v=Γ_{uuu}g^{uv}+Γ_{uuv}g^{vv} = -\frac{4v}{Δa^2b^2}$
$Γ_{uv}^u=0$
$Γ_{uv}^v=0$
$Γ_{vv}^u=Γ_{vvu}g^{uu}+Γ_{vvv}g^{vu} = -\frac{4u}{Δa^2b^2}$
$Γ_{vv}^v=\frac{4v}{Δb^4}$

Equation de Gauss

$\frac{\pd^2 F}{\pd u^1\pd u^2}=L_{ij}n+Γ_{ij}^ke_k
$normal tangentielle

\section{Arc de Surface}
Vecteur tangent :$\dv{s} F(u(s)) = \pdv{F}{u^1}(u(s)) \dot u^1 (s)+ \pdv{F}{u^2}(u(s)) \dot u^2 (s)= \dot u^i(s) e_j(u(s))$
Les composantes du vecteur tangent dans les base locale $\{e_1,e_2\}$ sont $\dot u^1(s)$ et $\dot u^2(s)$. La normale du vecteur tangent est: $\norm{\dv{s} F(u(s))}^2=\expval{\dot u^i e)i, \dot u^je_j} = \dot u^i(s)\dot u^j(s)\expval{e_1(u(s)), e_j(u(s))} = g_{ij}(u(s)) \dot u^1 (s) \dot u^j(s)$

$s$ est long d'arc sur la courbe ssi $g_{ij}(u(s)) \dot u^1(s)\dot u^j(s)=1 \forall s$.

Son suppose que $s$ est longueur d'arc.

$\dv[2]{}{s} F(u(s)) = κ(s)N(s) $
courbure normale principale.

$κ(s)N(s)=\dv{s} g_{ij}(u(s) \dot u^i(s) \dot u^j(s) ) = \dv{s} \pdv{f}{u^i}(u(s)) \dot u^i(s)=\pdv{F}{u^i}{u^j}(u(s))\dot u^i(s)\dot u^j(s)+ \pdv{F}{u^j}(u(s))\ddot u^i(s)= (L_{ij}((j))n(s+Γ_{ij}^n(u(s)) e_n(u(s))\dot u^j(s)\dot u^j(s) +\ddot u^i(s)e_i(u(s))))$

$κ(s)N(s)=(\ddot u^k(s)+Γ_{ij}^n (u(s)\dot u^j(s)\dot u^j(s)) e_n(u(s)) +L_{ij}(u(s))\dot u^1(s)\dot u^(s) n(s(s))$
La courbure normale de la courbe est $κ_s(s)=L_{ij}(u(s))\dot u^i(s)\dot u^j(s)=κ(s) \expval{N(s), nnn}=κ(s)\cos(α(s))=L(T(s),T(s))$

La courbure géodésique de la courbure est la norme du vecteur $K_g=(\ddot u^k(s)+Γ_{ij}^k\dot u^1(s)\dot u^j(s))e_k(u(s))
κ_g(s)=\norm{K_g}=\sqrt{\expval{K_g,K_g}}$
\begin{definition}
	Une géodésique de la surface $M$ est une courbe sur $M$ dont la courbure géodésique est nulle.
	
	L'équation d'une géodésique est donc:
	$\ddot u^k(s)+Γ_{ij}^k(u(s))\dot u^i(s)\dot u^j(s)=0$
\end{definition}

La courbure normale d'une courbe paramètre par long d'arc est 
$$κ_n(s)=L(T(s),T(s))=L_{ij}(u(s))\dot u^i(s)\dot u^j(s)$$
\begin{theorem}
	La courbure normale d'une courbe régulière sur une surface régulière est donnée par:
	$$κ_n(t)=\frac{L(T(s),T(s))}{g(T(s),T(s))}$$
\end{theorem}
\begin{proof}
	$s=∫_{t_0}^t\norm{\dv{τ}F(u(τ))}\dd{τ}$
	
	$\dv{s}{t}=\sqrt{g_{ij}(u(t))\dot u^i(t)\dot u^j(t)}$
	$T(s)=\dv{s} F(u(t(s)))=\dv{t} F(u(t))|_{t=t(s)}\dv{t}{s}= \frac{\dot u^j(t) e_j(u(t))}{\sqrt{g_{ij}(u(t))\dot u(t)\dot u^j(t)}}|_{t=t(s)}=\frac{T(t)}{\sqrt{...}}$
	
	$κ_n(t(s))=L(\tilde T(s),\tilde T(s))=L(T(t(s))/\sqrt{...},T(t(s))/\sqrt{...})$ $\Rightarrow$
	$κ_n(t)=\frac{L(T(t(s)),T(t(s)))}{g(T(t), T(t))}$
\end{proof}

\begin{definition}
	La courbure normale de la surface M dans la direction $e\in T_pM$ est
	$κ_n(p)=\frac{L(e,e)}{g(e,e)}=\frac{L(e,e)}{\norm{e}^2}$
	
	La courbure normale d'une courbe sur M est la courbure normale de M dans la direction du vecteur tangent à le courbe.
\end{definition}

$κ_n(s)=K(s)\cosα(s),
α(s)=KK(n(u(s)),N(s))$
$ρ(s)=\frac 1{K(s)}=\text{ rayon de courbure de la courbe }$

$ρ_n(s)=\frac 1{κ_n(s)}= \text{ le rayon de courbure normale }
ρ(s)=ρ_n(s)\cos α(s)$

La courbe est géodésique ssi: $κ_g=0$ $\Leftrightarrow$ $K=κ_n$ $\Leftrightarrow$ $ρ=ρ_n$ $\Leftrightarrow$ $\cosα=1$ $\Leftrightarrow$ $N(s)$ et $n(u(s))$ sont colinéaire.

$T_pM$ est engendré par $T(s)$ et $n(u(s))\wedge T(s)$ $X(s)=F(u(s))$ $s$=long.d'arc $\expval{\dot x(s),\ddot(s)}=0$.
Donc la composante tangentielle de $\ddot x$ est colinéaire à $n(u(s))\wedge T(s)$
$κ_γ(σ)=\expval{\ddot x(s), n(u(s))\wedge T(s)}$

$\expval{a,[b\wedge]}=[a,b,c]$ produit triple $=\det[a b c]$
$κ_g(s)=[\ddot x(s), n(u(s)), \dot x(s)] = [\dot x(s), \ddot x(s), n(u(s))]$ si $s$ =long. d'arc.

Courbures normale et géodésique.
$κ_n=\frac{L(\dot x,\dot x)}{g(\dot x, \dot x)}=L(\dot x, \dot x)$, si $|\dot x|=1$

Courbure normale de M en p la direction $e\in T_0M$ est $\frac{L(e,e)}{g(e,e)}=κ_n(e)$.

$K_g = (\ddot u^i+R_{jk}^i\dot u^j\dot u^k)e_i si g_{iy}\dot u^i\dot u^j=1$, si $g_{ij}\dot u^1\dot u^j =1$. $κ_y = |Κ_y|$.

\begin{example}
	\begin{enumerate}
		\item Droit sur la selle $F(x,y)=\mqty(x\\y\\-\frac{x^2}{a^2}+\frac{y^2}{b^2})$
		Droite:
		\begin{align*}
			x(t) &= x_0 + αt\\
			y(t) &= y_0 + βt\\
			z(t) &= -\frac{x(t)^2}{a^2}+\frac{y(t)^2}{b^2}\\
			&= -\frac{x_0^2}{a^2}+\frac{y_0^2}{b^2} - (\frac{α^2}{a^2}-\frac{β^2}{b^2})t^2 - 2( -\frac{x_0α}{a^2}+\frac{y_0β}{b^2} )t
		\end{align*}
		Si $\frac{α^2}{a^2}-\frac{β^2}{b^2}=0$ alors
		$X(t) = \mqty(x(t)\\y(t)\\z(t))$
		est une droite sur $M$ passant par $(x_0, y_0, F(x_0,y_0))$. En chaque point de $M$  passant 2 droites sur $M$. $(α,β)$, $(α,-β)$.
		$X(t)=F(x(t),y(t))$
		$\dot X(t)= α\mqty(1\\0\\-2\frac{x(t)}{a^2})+β(0\\1\\2\frac{y(t)}{b^2})=e_1+e_2$
		Courbure normale:
		$κ_n(t)=\frac{L(\dot X(t,\dot X(t)}{g(\dot X(t,\dot X(t)}=0$
		(parce que $L=0$)
		\item Cercles sur la sphère
		$ρ=\sin ν$
		$α=π-ν$
		$κ=\frac 1ρ = \frac 1{\sin ν}$
		\begin{align*}
			κ_n&=κ\cos α\\
			&=-\frac{\sin ν}{\sin ν}=-1
		\end{align*}
		La courbure normale d'on cercle sur la sphère est -1. La courbure géodésique est donc:
		$K^2=κ_n^2+κ_g^2$
		$\frac 1{\sin^2ν}=1/κ_g^2$
		$\Rightarrow$ $κ_g^2 = (\ctg ν)^2$
		$κ_g=0$ $\Leftrightarrow$ $ν=\frac π2 \mod π$ $\Leftrightarrow$ le cercle est un grand cercle.
		Les grands cercles sont des géodésiques de la sphère.
	\end{enumerate}
\end{example}
\section{L'indicatrice de Dupin} % (fold)
\label{sec:l_indicatrice_de_dupin}
L'équation $L(ξ,ξ)=±1$; $L_{ij}ξ^iξ^j = ±1$
$L_{11}(ξ^1)^2 +L_{12}ξ^1ξ^2+L_{21}ξ^2ξ^1+L_{22}(ξ^2)^2=±1$
définit une conique :=  L'indicatrice de Dupin.

ellipse si $L$ est definie positive
parabole si $L≠0$ et $\det(L)=0$
hyperbole si L est indéfinie et $\det(L)≠0$.

On dit que $p\in M$ est un point:

elliptique si l'indic. de Dupin est une ellipse.
parabolique si ...
hyperbolique si ...
plat si $L_p=0$

\begin{example}
	\begin{enumerate}
		\item La sphère: $L=\mqty(1&0\\0&\sin^2 u) ξ^2 +\sin^2 νη^2=1$. L'indicatrice de Dupin est une ellipse $\Rightarrow$ tous les points de la sphère sont elliptiques.
		\item Le tore: $L=\mqty((a+b\cosβ)\cosβ& 0\\0 &b)$, $0<b<a$.
		$(a+b\cos β)\cos βξ^2+bη^2=±1$
		
		Si $\cos β>0$: le point de coord $(α,β)$ est elliptique.
		Si $\cos β=0$: le point est parabolique.
		Si $\cos β<0$: c'est un point hyperbolique.
		
	\end{enumerate}
\end{example}
Interprétation:
L'équation du translaté du plan tangent est: $(X-p)•n(p)=±ε$.
L'intersection de $M$ avec ce plan est décrit par:
$(F(u)-p)•n(p)=±ε$ Développement de Taylor de $F(u)$ an pt. $ u+uF(u)=p$. On peut supposer $F(0)=p$
$F(u)=F(0)+ \pdv{F}{u^1}(0)u^1+\pdv{F}{u^2}(0)u^2 +\frac 12\pdv{F}{u^1}{u^2}(0)u^iu^j+O(u^2)=p+u^1e_1+u^2e_2+\frac 12(L_{ij}n(p)+Γ_{ij}^ke_k)u^iu^j+O(u^3)$

$(F(u)-p)•n(0)=(u^1e_1+u^2e_2+\frac 12(L_{ij}n+Γ_{ij}^ke_k))•n +O(u^3)$
$=\frac 12 L_{ij}u^iu^j+O(u^3)=±ε$
$\Rightarrow$ $\frac 12 L_{ij}\frac{u^i}{\sqrt{ε}}\frac{u^j}{\sqrt{ε}} + O(..) = ±1$
$\frac 12 L_{ij}\frac{u^i}{\sqrt{ε}}\frac{u^j}{\sqrt{ε}} = ±1$

L'indic. de Dupin est les somme asymptotique lorsque $ε\to 0$ de l'intersection de $M$ avec son pan tangent translaté de $ε$.
\begin{example}
	Ellipse elliptique
	tore elliptique
	tore 
\end{example}
% section l_indicatrice_de_dupin (end)
\section{Géodésiques} % (fold)
\label{sec:geodesiques}
En géométrie euclidienne les segments se droites peuvent être caractérisés comme:
\begin{enumerate}
	\item des courbes de courbure nulle
	\item le plus court chemin entre deux points
	\item une courbe dont les vecteurs tangents sont parallèle les autres
\end{enumerate}

\begin{theorem}
	Soit $M$ une surface régulière et $(F,U,V)$ une carte p.t. $p\in V=F(U)$. Soit $z\in T_pM$ t.q. $|z|=1$, alors il existe une courbe géodésique sur $V$ passant par $p$ où son vect. tangent est $z$.
	De plus cette géodésique est unique.
	
	Elle est donné par la solution de l'équation diff-le
	$$\left\{\begin{array}{c}
		\ddot u(s)^i+Γ_{jk}^i(u(s))\ddot u(s)\dot u^k(s)=0\\
		F(u(0))=p\\
		\dot u^i(0)e_i(u(0))=z
	\end{array}\right.
	$$
	La géodésique est $F(u(s))$ où $s$ est une longueur d'arc.
\end{theorem}
\begin{proof}
	$(g)$ admet une unique solution pour toutes données initiales $p$ et $z\in T_pM$. Soit $s\mapsto u(s)$ cette solution et $X(s)=F(u(s))$ la courbe sur $M$ associée. Si $s$ est longueur d'arc de cette courbe, c.à.d. si $|\dot X(s)|=1$, alors sa courbure géodésique est $(\ddot u^i(s)+Γ_{ik}^i(u(s))\dot u^j(s)\dot u^k(s))e_i(u(s))=0$ $\Rightarrow$ c'est une géodésique.
	
	Pour démontrer le Thm il suffit de montrer que $|\dot X(s)|^2=1$ pour tout $s$.
	Pour $s=0$ on a:
	$\dot X(0)=\dv{s}F(u(s))|_{s=0}=\pdv{F}{u^1}(u(0))\dot u^1(0)+\pdv{F}{u^2}(u(0))\dot u^2(0)=\dot u^1(0)e_1(u(0))+\dot u^2(0)e_2(u(0))=z$
	$\Rightarrow$
	$|\dot X(0)|^2=|z|^2=1$
	On doit donc montrer que
	$\dv{s}|\dot X(s)|^2=0$.
	$\dv{s}|\dot X(s)|^2=\dv{s}\expval{\dot X(s),\dot X(s)}=2\expval{\dot X(s),\ddot X(s)}$
	
	$\dot X(s)=\dot u^i(s)e_1(u(s))$
	$\ddot X(s) = (\ddot u^i(s)+Γ_{jk}^i(u(s))\dot u^j(s)\dot u^k(s))e_i(u(s))+L_{ij}(u(s))n(u(s))\dot u^i(s)\dot u^j(s)$
	
	$\expval{\dot X(s),\ddot X(s)}=\expval{\dot u^l(s)e_l(u(s)), (\ddot u^i(s)+Γ_{jk}^i(u(s))\dot u_j(s)\dot u_k(s))e_i(u(s))}=\dot u^l(\ddot u^i+Γ_{jk}^i\dot u^j\dot u^k)\expval{e_l,e_i}= 0•g_{li}$
	$\Rightarrow$ $|\dot X(s)|^2=|\dot X(0)|^2=1$.
\end{proof}
\begin{example}
	Géodésiques se la sphère. toutes les courbes sur la sphère ont courbure normale $-1$: $L=-g$, $\frac Lg=-1$.
\end{example}
 $Κ^2=κ_g^2+κ_n^2 = κ_g^2 +1$ une courbe est géodésique sur la sphère si sa courbure est $1$
 $\ddot X=Κ n=κ_nn=-n=-X$
$ X(s)=X(0)\cos(s)+\dot X(0)\sin(s)$
$X(s)$ est le grand cercle intersection de la sphère avec le plan dirigé pur $X(0)$ et $\dot X(0)$.
% section geodesiques (end)
\section{Changement de Coordonnées} % (fold)
\label{sec:changement_de_coordonnees}
$Φ:F_1^{-1}(V_1\cap V_2)\rightarrow F_2^{-1}(V_1\cap V_2)$ est un difféomorphisme.

$e_1(u)=\pdv{F_1}{u^1}$
$e_2(u)=\pdv{F_1}{u^2}$
$\bar e_1(v)=\pdv{F_1}{v^1}$
$\bar e_2(v)=\pdv{F_1}{v^2}$

$F_2ºΦ=F_1 φ(u)=\mqty[Φ_1(u_1,u_2)\\Φ_2(v_1,v_2)]=\mqty[v_1\\v_2]$

$e_1(u)=\pdv{u^1} F_2ºΦ(u)$
$=\pdv{F_2}{v^1}ºΦ(u)º\pdv{Φ_2}{u^1}+\pdv{F_2}{v^2}ºΦ(u)º\pdv{Φ_2}{u^1}$
$=\bar e_1ºΦ\pdv{Φ_1}{u_1}+\bar e_2ºΦ\pdv{Φ_2}{u^1}$
$=\pdv{v_1}{u^1}\bar e_1(v) + \pdv{v_2}{u^1}\bar e_2(v)=\pdv{v^i}{u^1}\bar e_i(v)$

$e_2(u)=...=\pdv{v^i}{u^2}\bar e_i(v)$

Si $X\in T_{F_1(U)}M=T_{F_2(V)}M$
$X=X^je_j(u)=\bar X^i\bar e_i(v)$

$\bar X^i \bar e_i(v)=X^j\pdv{V^i}{u^j}\bar e_i(v)$

donc $\bar X^i=X^j\pdv{V^i}{u^j}$

$\bar X^i=T_j^i X^j$

Matrice de passage
$T_j^i=\pdv{V^i}{u^j}=\pdv{Φ_i}{u^j}=$ matrice Jacobienne de $Φ = DΦ=Φ'$.

$g_{ij}(u)=\expval{e_i(u),e_j(u)}$
$\bar g_ij(v)=\expval{\bar e_i, \bar e_j(v)}$
$\Rightarrow$ $g_{ij}(u)=\expval{\pdv{v^k}{u^i}\bar e_k(v),\pdv{v^l}{u^j}\bar e_l(v)}=\pdv{v^k}{u^i}\pdv{v^l}{u^j}\expval{\bar e_k(v),\bar e_l(v)}= \pdv{v^k}{u^i}\pdv{v^l}{u^j}\bar g_{kl}(v)$

$\bar g_{ij}(v)=\pdv{u^k}{v^i}\pdv{u^l}{v^j}g_{kl}(u)
$ $\pdv{u^k}{v^i}$ = matrice jacobienne de $Φ^{-1}=DΦ^{-1}=(Φ^{-1})'=(DΦ)\dmoºΦ\dmo=(Φ')\dmoºΦ\dmo =\bar T_i^k(v)$

$Φ\dmoºΦ=Id$  
$(Φ\dmo)'ºΦ Φ'=Id$
$\bar T_i^k(v) T_j^i(u)=δ_j^i$
$\bar T(v)=T(u)\dmo$

$T_{j_1j_2...j_n}^{i_1i_2...i_k}$ est un tenseur $k$ fois contravariant et $n$ fois covariant si il se transforme comme:
 
$$\bar T_{j'_1j'_2...j'_n}^{i'_1i'_2...i'_k}(v)=T_{j_1j_2...j_n}^{i_1i_2...i_k}•\pdv{u^{j_1}}{v^{j'_1}}...\pdv{u^{j_n}}{v^{j'_n}}•\pdv{v^{i'_1}}{v^{i_n}}...\pdv{v^{i'_k}}{u^{i_k}}$$

Exemple: La 1ère forme fondamentale est un tenseur 2 fois covariant. Un champ de vecteur tangent sur M est un tenseur 1 fois contravariant.

La 2ème forme fondamentale:
$$L_{ij}(u)=\expval{\pdv{F_1}{u^i}{u^j}(u), n(u)}
$$$$\bar L_{ij}(v)=\expval{\pdv{F_2}{u^i}{u^j}(u), n(v)}
$$
$$F_2ºΦ=F_1$$

$$L_{ij}(u)=\expval{\pdv{F_2}{u^i}{u^j}ºΦ(u),n(u)}=\expval{\pdv{u^i} (\pdv{v^k}{u^j}\pdv{F_2}{v^k}ºΦ(u)),n(u)}=\expval{\pdv{v^k}{u^i}{u^j}\bar e_k(v)+\pdv{v^k}{u^j}\pdv{F}{v^k}{v^l}ºΦ(u)\pdv{v^l}{u^i},n(u)}=\pdv{v^k}{u^j}\pdv{v^l}{u^1}\expval{\pdv{F}{v^k}{v^l}(v),n(v)}=\pdv{v^k}{u^j}\pdv{v^l}{u^i}\bar L_{kl}(v)$$

La 2ème forme fondamentale est un tenseur $2$ fois covariant.
Si $X$ et $Y$ $\in T_{F(u)}M=T_{F(v)}M$
$L(X,Y)=L_{ij}(u)X^iY^j=\bar L_{kl}(v)\pdv{v^k}{u^i}\pdv{v^l}{u^j}X^iY^j=\bar L_{kl}(v)(\pdv{v^k}{u^i}X^i)(\pdv{v^l}{u^j}Y^j)=\bar L_{kl}(v)\bar X^k\bar Y^j$

$g(X,Y)=\expval{X,Y}=g_{ij}(u)X^iY^j=\bar g_{kl}(v)\bar X^k\bar Y^l$

$(g^{ij}(u))=(g_{ij}(u))\dmo$

$g_{ij}=\bar g_{kl}\pdv{v^k}{u^j}\pdv{v^l}{u^i}$
$g=(DΦ)^{Τ}\bar g(DΦ)$

$g\dmo=(DΦ)\dmo\bar g\dmo (DΦ^Τ)\dmo$

$g^{ij}=\pdv{u^i}{v_k}\bar g^{kl}\pdv{u^j}{v^l}$

$g^{ij}$ est un tenseur 2 fois contravariant.

Weingarten: $W_j^i(u)=-\pdv{n^i}{u^j}$ est un tenseur 1 fois covariant et 1 fois contravariant.

$Γ_{ij}^k$ n'est pas un tenseur!
 
% section changement_de_coordonnees (end)
\section{Géodésiques} % (fold)
\label{sec:geodesiques}
\begin{enumerate}
	\item Courbure nulle $\Rightarrow$ courbure géodésique nulle.
	\item Courbe le longueur minimale entre 2 pts. $\Rightarrow$ courbe de long. Extrémale parmis toutes les courbes reliant 2 points. 
	\item vecteurs tangents tous parallèle $\Rightarrow$ vecteurs tangents parallèle sur la courbe.
\end{enumerate}
Longueur d'une courbe sur $M$:
$c:[a,b]\ni t\mapsto F(u(t))=c(t)\in M$.

$\mathcal{L}=\mathcal{L}[a]=∫_a^b\underbrace{\sqrt{g_{ij}(u(t))\dot u^i(t)\dot u^j(t)}}_{\norm{\dot c(t)}}\dd{t}$

$c(a)=A$, $c(b)=B$.

$\min\mathcal{L}[a]=\mathcal{L}_{\text{min}}$

$$\left\{\mqty{u:[a,b]\rightarrow U\\F(u(a))=A\\F(u(b))=B}\right.$$

Si $\bar u:[a,b]\rightarrow U$ t.q. $F(\bar u(a))=A$ et $F(\bar u(b))=B$ Satisfait $\mathcal{L}[\bar u]=\mathcal{L}_{\text{min}}$ alors
$$\dv{λ} \mathcal{L}[\bar u+λv]|_{λ=0}=0$$

Pour tout $v:[a,b]\rightarrow U$ t.q. $v(a)=v(b)=0$. $\mathcal{L}[u]=∫_a^b L(u(t),\dot u (t))\dd{t}$
où $L(u,w)=\sqrt{g_{ij}(u)w^iw^j}$

$\mathcal{L}[u+λv]=∫_a^b L(u(t)+λv(t),$$\dot u(t)+λ\dot v(t))\dd{t}$.

$\dv{λ} \mathcal{L}[u+λv]|_{λ=0}=∫_a^b [\pdv{L}{u}(u(t),\dot u(t))v(t)+\pdv{L}{w}(u(t)(\dot u(t))\dot u (t))\dot v(t)]\dd{t}=$
$\pdv{L}{w}(u(t),\dot u(t))v(t)|_a^b + ∫_a^b [\pdv{L}{u}(u(t),\dot u(t))-\dv{t}\pdv{L}{w}(u(t),\dot u(t))]v(t)\dd{t}=∫_a^b[...]v(t)\dd{t}=0$.

$\Rightarrow$ Equation d'Euler-Lagrange:

$$ \pdv{L}{u} (u(t),\dot u(t)=\dv{t}\pdv{L}{w} (u(t),\dot u(t))\&\& \forall t\in [a,b]$$

Avec $L(u,w)=\sqrt{g_{ij}w^iw^j}$ on a:
$\pdv{L}{u^k}(u,w)=\frac 12$ $\frac{g_{i_1k}(u)w^iw^j}{\sqrt{g_{ij}(u)w^iw^j}}$

$\pdv{L}{w^k}(u,w)=\frac 12  \frac{g_{ij}(u)(δ_k^i w^j+w^iδ_k^j)}{\sqrt{...}}=\frac 12 \frac{g_{kj}w^2+g_{ik}w^i}{\sqrt{...}}=\frac{g_{ki}w^i}{\sqrt{...}}$.

On suppose que la courbe $\bar u$ est paramétrée par longueur d'arc
$\sqrt{g_{ij}(\bar u(t))\bar\dot u^i(t)\dot\bar u^j(t)}=1$

$\pdv{L}{u^k}(u(t),\dot\bar u(t))=\frac 12 g_{ij_1k}(u(t))\dot\bar u^i\dot\bar u^j(t)$

$\pdv{L}{w^k}(\bar u(t),\dot\bar u(t))=g_{ki}(u(t))\dot\bar u^i(t)$

$\dv{t} \pdv{L}{w^k}(\bar u(t),\dot\bar u(t))=g_{ki,j}(\bar u(t)\dot\bar u^j(t)\dot\bar u^i(t)+g_{ki}(\bar u(t))\ddot\bar u^i(t))$

$\frac 12 g_{ij,k}(\bar u(t))\dot\bar u^i(t)\dot\bar u^j(t)=g_{ki,j}(\bar u(t))\dot\bar u^j(t)\dot\bar u^i(t) +g_{ki}(\bar u(t))\ddot\bar u^i(t)=g_{ki}(\bar u(t))\ddot\bar u^i(t)+\underbrace{\frac 12(g_{ki,j}(\bar u(t))+g_{kj,i}(\bar u(t))-g_{ij,k}(\bar u(t)))}_{Γ_{ijk}(\bar u(t))}\dot\bar u^i(t)\dot\bar u^j(t)=0$

$∑_{ij}a_{ij}b_{ij}=∑_{ij}\frac 12 (a_{ij}+a_{ij})b_{ij}=\frac 12 ∑_{ij}a_{ij}b_{ij}+\frac 12 ∑_{ij}a_{ji}b_{ij}=∑_{ij}a_{ij}\frac 12(b_{ij}+b_{ij})$

$g^{kl}(\bar u(t))[g_{ki}(\bar u(t))\ddot\bar u^i(t)+T_{ijk}(\bar u(t))\dot\bar u^i(t)\dot\bar u^j(t)]=0$

$\ddot\bar u^l(t)+Γ_{ij}^l(\bar u(t))\dot\bar u^i(t)\dot\bar u^j(t)=0$

$\Rightarrow$ $c(t)=F(\bar u(t))$ est géodésique! 

\begin{theorem}[III.19]
	Une courbe sur la surface $M$ relient $A\in M$ à $B\in M$ est géodésique ssi sur longueur est extrémale parmis les courbes sur $M$ relient $A$ à B.
\end{theorem}

\begin{example}
	\begin{enumerate}
		\item Sphère
		\item Cylindre
	\end{enumerate}
\end{example}

% section geodesiques (end)
\section{Dérivée  directionnelle et covariante} % (fold)
\label{sec:derivee_directionnelle_et_covariante}
\begin{definition}
	$TM=\{(p,v)|p\in M, v\in T_p M\}$ Fibré tangent à $M$. $v$ un champ de vecteur sur $M$ est une application lisse $v:M\rightarrow TM$ t.q. $v(p)\in T_pM$.
\end{definition}

Si $[a,b]\ni t\mapsto c(t)\in M$ est une courbe sur $M$, un champ de vecteurs sur $c$ est une application lisse $[a,b]\ni t\mapsto v(t)\in T_{c(t)}M$.

En coord locales un champ de vect. sur $M$ est décrit par $V(F(u))=V^i(u)e_i(u)$ où la fonctions $V^i(u)$ sont lisses. On champ sur les courbe $c(t)=F(u(t))$ est définit par:
$v(t)=v^i(t)e_i(F(u(t)))$ les fonction $v^1(t)$ et $v^2(t)$ e'tant lisses.

Si $v$ est un champ de vecteurs sur $M$, alors les $v^i(u)$ forment un tenseur 1 fois contravariant.

$v(F(u))=v^i(u)e_i(u)=\bar u^j(v)\bar e_j(\bar v)=V(G(v))$.

Si $F(u)=G(v)$.

$e_i(u)=\pdv{F}{u'}(u)=\dv{u'}$ $G(v(u))=\dv{G}{v^j}(v(u))\dv{V_j}{u^i}(u)=\bar e_j(v)\pdv{V_j}{u^i}$

$V^i(u)\pdv{V_j}{u^i}\bar e_j(v)=\bar v^j(v)\bar e_j(v)$

$v^i(u)\pdv{V_j}{u^i}=\bar v^j(v)$.

\begin{example}
	\begin{enumerate}
		\item 	Sur la sphère:
	$F(v,φ)=\vc{\sin v\cos φ\\\sin v\sin φ\\ \cos v}$
	$e_v(u,v)=\vc{\cos v\cos φ\\\cos v\sin φ\\-\sin v} $définit un champ de vecteur sur la sphère sauf aux pôles.
	\item Le gradient d'une fonction.
	$f:M\rightarrow \R$
	$d_pf:T_pM\rightarrow \R, X\mapsto d_pf(X)$ dérivée de $f$ en $p$, application linéaire

	$d_pf(X)=\dv{t}f(c(t))|_{t=0}$ où $c(t)$ est une courbe sur M t.q. $c(0)=p$ et $\dot c(0)=X$
	$d_pf(X)=\dv{t}fºF(u(t))|_{t=0}=\dv{t}\tilde f(u(t))|_{t=0}=\pdv{\tilde f}{u^1}(u(0))\dot u^1(0)+\pdv{\tilde f}{u^2}\dot u^2(0)=\pdv{\tilde f}{u^1}(u(0))X^1+\pdv{\tilde f}{u^2}(u(0))X^2=\expval{\nabla_pf,X}=g_{ij}(u(0))(\nabla_pf)^iX^j$
	
	$(\nabla_pf)^i=g^{ik}(u(0))\pdv{\tilde f}{u^k}(u(0))$
	
	$\underbrace{g_{ji}g^{ik}}_{δ_j^k}\pdv{\tilde f}{u^k}X^j=\pdv{\tilde f}{u^j}x^j$
	
	
	le gradient d'une function. En cord. locale, le gradient d'une function $f$ est le champ de vecteur
	$(\nabla f)^i=g^{ij}\pdv{\tilde f}{u^j}$
	$(\nabla f)(F(u))=g^{ij}(u)\pdv{\tilde f}{u^j}(u)e_i(u)$
	
	$\vc{(\nabla f)^1\\(\nabla f)^1}=g\dmo\vc{\pdv{f}{u^1}\\\pdv{f}{u^2}}$
	
	Les $(\nabla f)^i$ forment un tenseur 1 fois contravariant. Les dérivée partielles $\pdv{\tilde f}{u^j}=f_{,j}$
	
	Se transforment selon:
	
	$\tilde f(u)=\bar\tilde f(v)$
	
	$\pdv{\tilde f}{u^i}(u)=\pdv{\bar\tilde f}{v^j}(v(u))\pdv{v^j}{u^i}$
	
	ce sont les composantes d'un tenseur 1 fois covariant.
	\end{enumerate}
	
	
\end{example}
\begin{definition}
	La dérivée directionnelle d'une fonction $f:M\rightarrow \R$ en $p\in M$ dans en direction $X\in T_pM$ est $\pd_Xf(p)=d_pf(X)=\expval{\nabla_pf,X}$
\end{definition}
\begin{lemme}
	Si $X,Y\in T_pM$, il existe un unique $Z\in T_pM$, tel que pour toute fonction $f:M\rightarrow \R$:
	$\pd_x\pd_yf(p)-\pd_x\pd_yf(p)=\pd_zf(p)$.
	On note $Z=[X,Y]$. $Z$ est la Crochet de Lie de $X$ et $Y$.
\end{lemme}
\begin{proof}
	Localement:
	$X=X^ie_i$
	$Y=Y^je_j$
	$p=F(0)$
	$\pd_Xf(p)=X^i(0)f_{,i}(0)$
	$\pd_Yf(p)=Y^j(0)f_{,j}(0)$
	$\pd_X\pd_Yf(p)=X^i(0)\pdv{u^i}(\pd_Yf)(F(u))=X^i(0)\pdv{u^i} Y^j(u)\pdv{f}{u^j}(u)|_{u=0}=X^i(0)Y^j_{,i}(0)\pdv{f}{u^j}(0)+X^i(0)Y^j(0)\pdv{f}{u^i}{u^j}(0)$
	
	$\pd_X\pd_Yf-\pd_Y\pd_Xf=X^i(0)Y^j_{,i}(0)\pdv{f}{u^j}(0)-Y^i(0)X^j_{,i}(0)\pdv{f}{u^j}(0)=X^j\pdv{f}{u}(0)$
	
	$Z^j=X^i(0)Y^j_{,i}(0)-Y^i(0)X^j_{,i}(0)$
	
	$|box [X,Y](F(u))=(X^i(u)Y^j_{,i}(u)-Y^i(u)X_{,i}^j(u)e_j(u))|$
\end{proof}

\begin{remark}
	
	$$\left\{\mqty([λX+μY, Z]=λ[X,Z]+μ[Y,Z]\\	[X,Y]=-[Y,X]\\  Jacobi: [X,[Y,Z]]+[Y,[Z,X]]+[Z,[X,Y]]=0)\right.$$
	
	L'ensemble des champs de vect . sur M est une algèbre de Lie pour $[•,•]$ Notation Moderne
	
	$X=X^i(u)\pdv{u^i}$
	
	$e_i(u)=\pdv{u^i}$
	
\end{remark}


\begin{definition}
	Pour $p\i M$, soit $π_p$ la projection orthogonal de $\R^3$ sur $T_pM$ la dérivée covariante d'un champ de vecteurs $X$ sur une courbe $c:[a,b]\rightarrow  M$ est défini comment 
	$\frac{DX}{\dd{t}}(t)=π_{c(t)}\dv{X}{t}(t)$
\end{definition}
\begin{example}
	\begin{enumerate}
		\item Courbe plane $\Rightarrow$ n=const
		$Y(t)\in\mathcal{P}$ $\Rightarrow$ $\forall t \dot Y(t)=\lim_{ε\downarrow 0}\frac{Y(t+ε)-Y(t)}{ε}\in\mathcal{P}$ $\Rightarrow$ $\dot Y(t)=\frac{DY}{\dd{t}}(t)$.
		\item Sur la sphère.
		$c(t)=(\cos t\sin θ,\sin t\sin θ,\cos θ)$
		
		$\frac{D\dot c}{\dd{t}}(t)=e_φ(t,θ)$
		
		$\dot c(t)=\vc{-\sin t\sin θ\\\cos t\sin θ\\ 0}$
		\item 
		$\dv{t}\dot c(t)=\ddot c(t)=\vc{-\cos t\sin θ\\-\sin t\sin θ\\0}$
		$n(θ,φ)=F(θ,φ)=\vc{\cos φ\sin θ\\\sin φ\sin θ\\\cos θ}$
		
		$\frac{D\dot c}{\dd{t}}(t)=π_{c(t)}\ddot c(t)=\ddot c(t)-n(θ,t)\expval{n(θ,t),\ddot c(t)}=\vc{-\cos t\sin θ\\-\sin t\sin θ\\0}-\vc{\cos t\sin θ\\\sin t\sin θ\\\cos θ}(-\sin^2θ)=-(1-\sin^2θ)\vc{\cos t\sin θ\\\sin t\sin θ\\ 0}+\cos θ\sin^2 θ\vc{0\\0\\1}$.
		
		$\frac{D\dot c}{\dd{t}}(t)=0$ $\Leftrightarrow$ $\sin 2θ=0$ $\Leftrightarrow$ $θ=0$ où $θ=\frac π2$
		
		Les cercles sur la sphère t.q. $\frac{D\dot c(t)}{\dd{t}}=0$ sont les grand cercles (les géodésiques!).
		
		$\frac{DX}{\dd{t}}(t)=π_{c(t)}\dv{X}{t}(t)=(\dot X^k(t)+Γ_{ij}^k(u(t))X^i(t)\dot U^j(t))e_k(u(t))$.
	\end{enumerate}
\end{example}

\begin{definition}
	Soit $X$ un champ de vecteur sur $M$ et $Y\in T_pM$. La dérivée covariante du champ $X$ en $p$ dans la direction $Y $est
	$$ (\nabla_YX)(p)=\frac{DX(c(t))}{\dd{t}}|_{t=0}$$ où $c$ est une courbe sur $M$ tq. $c(0)=0$ et $\dot c(0)=Y$.
\end{definition}

\begin{remark}
	$\frac{DX(c(t))}{\dd{t}}|_{t=0}$ ne dépend que de $c(0)$ et $\dot c(0)$.
	
	En coordonnes locale on a: $c(t)=F(u(t))$ avec $F(u(0))=p$ et $\dot c(t)=\dot u^i(0)e_i(u(0))=Y$
	$\dv{t} X(c(t))=\dv{t}X(F(u(t))=\dv{t} X^i(u(t))e_i(u(t))$
	$\Rightarrow$ $\dv{t}X^i(t)=\pdv{X^i}{u^j}\dot u^j(t)$
	$(\nabla_YX)(p)=(X_{ij}^k(u(0))\dot u^j(0)+Γ^k_{ij}(u(0))X^i\dot u^j(0))e_k(u(0))$
	
	$\nabla_YX=(X_{ij}^k+Γ_{ij}^kX^i)\dot u^j=(X^k_{ij}+Γ_{ij}^kX^i)Y^j)e_k=(\pd_YX^k+Γ_{ij}^kXY^j)e_k$.
\end{remark}
\begin{property}
	$λ,μ\in \R; X,X_1,X_2,Y,Y_1,Y_2$ champs de vecteurs $f:M\rightarrow \R$.
	
	$\nabla_Y(λX_1+μX_2)=λ\nabla_YX_1+μ\nabla_YX_2$
	$\nabla_Y(fX)=(\dd_Y f)X+f\nabla_YXe_k(u(0))$
	
	$\dd_Y\expval{X_1,X_2}=\expval{\nabla_YX_1,X_2}+\expval{X_1,\nabla_YX_2}$
	$\nabla_{λY_1+μY_2}X=λ\nabla_{Y_1}X+μ\nabla_{Y_2}X$
	$\nabla_{fY}X=f\nabla_YX$.
\end{property}
\begin{example}
	Sur la sphère  $\nabla_YX=(\nabla_YX)^θe_θ+(\nabla_YX)^φe_φ$
	
	$(\nabla_YX)^θ=(X_θ^θ+Γ_{θθ}^θ)Y^θ+Γ_{φθ}^θX^φY^θ+(Χ_φ^θ+Γ_{θφ}^θX^θ)Υ^φ+Γ_{φφ}^θX^φY^φ=Χ_{,θ}^θY^θ+X_{,φ}Υ^φ-\frac 12 \sin(2θ)X^φΥ^φ$
	
	$(\nabla_YX)^φ=X_{,θ}^φY^θ+X_{,φ}^φ+\ctg(θ)(X^φΥ^θ+X^θY^φ)$
\end{example}
% section derivee_s_directionnelle_et_covariante (end)